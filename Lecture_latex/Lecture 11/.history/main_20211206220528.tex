\documentclass{beamer}
\usepackage[utf8]{inputenc}

\usetheme{Madrid}
\usecolortheme{default}
\useinnertheme{circles}

\definecolor{Logo1}{rgb}{0.208, 0.2865, 0.373}
\definecolor{Logo2}{rgb}{0.000, 0.674, 0.863}

\setbeamercolor*{palette primary}{bg=Logo1, fg=white}
\setbeamercolor*{palette secondary}{bg=Logo2, fg=white}
\setbeamercolor*{palette tertiary}{bg=white, fg=Logo1}
\setbeamercolor*{palette quaternary}{bg=Logo1,fg=white}
\setbeamercolor{structure}{fg=Logo1} % itemize, enumerate, etc
\setbeamercolor{section in toc}{fg=Logo1} % TOC sections

\usepackage{graphicx,animate}
%------------------------------------------------------------
%This block of code defines the information to appear in the
%Title page
\title[Linear Algebra] %optional
{Matrix Diagonalization; Complex Matrix}

\subtitle{Lecture 11}

\author[11910803@mail.sustech.edu.cn] % (optional)
{
    Zhang Ce
}

\institute[] % (optional)
{
    Department of Electrical and Electronic Engineering\\
    Southern University of Science and Technology
}

\date[2021.12.7] % (optional)
{2021.12.7}


%End of title page configuration block
%------------------------------------------------------------



%------------------------------------------------------------
%The next block of commands puts the table of contents at the
%beginning of each section and highlights the current section:

\AtBeginSection[]
{
\begin{frame}
    \frametitle{Table of Contents}
    \tableofcontents[currentsection]
\end{frame}
}
%------------------------------------------------------------


\begin{document}

%The next statement creates the title page.
\frame{\titlepage}


%---------------------------------------------------------
%This block of code is for the table of contents after
%the title page
\begin{frame}
\frametitle{Table of Contents}
\tableofcontents
\end{frame}
%---------------------------------------------------------
\section{A Brief Review of Last Lecture}
\begin{frame}{Last Lecture, We Discuss\dots}
Three parts in last lecture:
    \begin{enumerate}
        \item Computing Techniques of Determinants\\
        review for type 1-5; new: type 6
        \item Eigenvalues and Eigenvectors\\
        difinition; meaning of "eigen"; geometrical interpretation; calculation; two formulas: trace and determinant
        \item Diagonalization of Matrix\\
        deduction; example; a deeper view; special: symmetric matrix
        \item Important Applications of Eigenvalues and Eigenvectors\\
        Google: PageRank algorithm; principle component analysis(PCA)
    \end{enumerate}

\end{frame}

\begin{frame}{Type 6: In-Order Matrix}
    Compute the nth order determinant:
    \begin{equation*}
        \det A=\left| \begin{matrix}
        1+{x_1}^2&		x_1x_2&		\cdots&		x_1x_n\\
        x_2x_1&		1+{x_2}^2&		\cdots&		x_2x_n\\
        \vdots&		\vdots&		\ddots&		\vdots\\
        x_nx_1&		x_nx_2&		\cdots&		1+{x_n}^2\\
    \end{matrix} \right|
    \end{equation*}

    \begin{equation*}
        \det A=\det \left( I+\left[ \begin{array}{c}
            x_1\\
            x_2\\
            \vdots\\
            x_n\\
        \end{array} \right] \left[ \begin{matrix}
            x_1&		x_2&		\cdots&		x_n\\
        \end{matrix} \right] \right)
    \end{equation*}

Hint: Using eigenvalue formula for determinants:
\begin{equation*}
\lambda _1\lambda _2\cdots \lambda _n=\det A
\end{equation*}
Eigenvalues are $1, 1+u^Tu$. Multiply all of them, $\det A=1+\sum_{i = 1}^{n} x_i^2 $.
\end{frame}

\begin{frame}{Understanding Eigenvalues in Geometry}
    Find the eigenvalues and eigenvectors for these matrix.

    \begin{enumerate}
        \item $\left[ \begin{matrix}
            1&		0\\
            0&		1\\
        \end{matrix} \right]$, the identity matrix (how many eigenvectors?)
        \item $\left[ \begin{matrix}
            1&		1\\
            0&		1\\
        \end{matrix} \right]$, "shear" matrix
        \item $\left[ \begin{matrix}
            1&		0\\
            0&		0\\
        \end{matrix} \right]$, projection (onto $x$ axis) matrix
        \item $\left[ \begin{matrix}
            0&		0\\
            0&		0\\
        \end{matrix} \right]$, zero matrix
        \item A challenging one: $A=uu^T$, a rank 1 matrix\\
        Hint: you may start with projection (onto $u$) matrix
    \end{enumerate}
    \end{frame}

\begin{frame}{Two Formulas: Trace and Determinant}
$\det(A-\lambda I)$ is a polynomial of $\lambda$. For an $n\times n$ matrix, $\det(A-\lambda I)$ is a $n$ degree equation with only 1 unknown.

\vspace{3pt}
Suppose $n=2$:
\begin{equation*}
    \left| \begin{matrix}
        x_{11}-\lambda&		x_{12}\\
        x_{21}&		x_{22}-\lambda\\
    \end{matrix} \right|=\lambda ^2-\left( x_{11}+x_{22} \right) \lambda +\left( x_{11}x_{22}-x_{12}x_{21} \right)
\end{equation*}

Adopt Vieta's Theorem:
\begin{equation*}
    \lambda _1+\lambda _2=x_{11}+x_{22},\:\: \lambda _1\lambda _2=x_{11}x_{22}-x_{12}x_{21}
\end{equation*}

\vspace{3pt}
Pure algebra: higher-order Vieta's Theorem gives us:

\begin{equation*}
    \lambda _1+\lambda _2+\cdots +\lambda _n=trace\left( A \right) ,\:\: \lambda _1\lambda _2\cdots \lambda _n=\det A
\end{equation*}
\end{frame}

\begin{frame}{Diagonalization of Matrix}
You have found all the eigenvectors and eigenvalues of a matrix, but\dots

\vspace{3pt}
Experts in Linear Algebra will not stop here, please find matrix expression.

\vspace{3pt}
\textbf{Condition: $n\times n$ matrix $A$ has $n$ linearly independent eigenvectors.} Then we write the $n$ linearly independent eigenvectors in the columns of a matrix $P$ (We have done this many times in this course...). Magical matrix multiplication gives us:
\begin{equation*}
    AP=P\varLambda
\end{equation*}
\begin{equation*}
    A\left[ \begin{matrix}
        |&		|&		|&		&		|\\
        |&		|&		|&		&		|\\
        x_1&		x_2&		x_3&		\cdots&		x_n\\
        |&		|&		|&		&		|\\
        |&		|&		|&		&		|\\
    \end{matrix} \right] =\left[ \begin{matrix}
        |&		|&		|&		&		|\\
        |&		|&		|&		&		|\\
        x_1&		x_2&		x_3&		\cdots&		x_n\\
        |&		|&		|&		&		|\\
        |&		|&		|&		&		|\\
    \end{matrix} \right] \left[ \begin{matrix}
        \lambda _1&		&		&		&		\\
        &		\lambda _2&		&		&		\\
        &		&		\lambda _3&		&		\\
        &		&		&		\cdot&		\\
        &		&		&		&		\lambda _n\\
    \end{matrix} \right]
\end{equation*}

If you use column method of matrix multiplication, that matrix equation is easy to verify.
\begin{equation*}
    A=P\varLambda P^{-1}
\end{equation*}
\end{frame}

\begin{frame}{Matrix Diagonalization: A Deeper View}
    \begin{equation*}
        \left[ \begin{matrix}
            2&		1\\
            1&		2\\
        \end{matrix} \right] =\left[ \begin{matrix}
            1&		1\\
            -1&		1\\
        \end{matrix} \right] \left[ \begin{matrix}
            1&		0\\
            0&		3\\
        \end{matrix} \right] \left[ \begin{matrix}
            1&		1\\
            -1&		1\\
        \end{matrix} \right] ^{-1}
    \end{equation*}

Suppose we have a input coordinates $(2,0)$ in standard basis, how can we find the output?
\begin{figure}
    \centering
    \includegraphics[width=\textwidth]{PNP-1.jpg}
\end{figure}

Essence: change basis and simplify the linear transformation matrix.
\end{frame}

\begin{frame}{Important Geometrical Interpretation}
    \begin{equation*}
        A=P\varLambda P^{-1}
    \end{equation*}
\begin{enumerate}
    \item $P$ is a "change of basis" matrix.
    \item $A$ and $\Lambda $ are the same linear transformation using different basis.
    \item Diagonalization: change to eigenvector basis, make linear transformation $A$ a simple stretching transformation $\Lambda$.
\end{enumerate}

If you can understand all of these, chapter 5 is simple combination of previous knowledge. Remember, always keep geometrical interpretation in head (just as we do when learning Gram-Schmidt process and vector spanning), instead of memorizing the formula.
\end{frame}

\section{Matrix Diagonalization}
\begin{frame}{Core Problems of Matrix Diagonalization}
    \begin{equation*}
        A=P\varLambda P^{-1}
    \end{equation*}
\textbf{Problems:}
\begin{enumerate}
    \item How to tell whether a matrix is diagonlizable or not?
    \item How to find a diagonalization matrix $P$?
\end{enumerate}

\textbf{Answers:}\\
\vspace{3pt}
For problem 1, we know that a matrix is diagonlizable if and only if it has $n$ linearly independent eigenvectors.

\vspace{3pt}
For problem 2, we need to find all the eigenvalues and eigenvectors. The eigenvalues will go into $\varLambda$ and eigenvectors will go into $P$.

\vspace{5pt}
If a matrix is diagonlizable, we can easily find the diagonalization matrix! But the question is, we cannot tell if a matrix has $n$ linearly independent eigenvectors (i.e. if it is diagonlizable).
\end{frame}

\begin{frame}{Another Simplified Criteria}
If a $n\times n$ matrix has $n$ distinct eigenvalues, then it must be diagonlizable.

\vspace{3pt}
Why? I will not give you the proof, but I want you to understand. I will show from 2 sides:

\begin{enumerate}
    \item For each eigenvalue, there must be at least 1 eigenvector (authentic to say: 1-dimensional).
    \item For the same eigenvector (all vectors that in the 1-dimensional line), it cannot correspond to 2 eigenvalues.\\
    (Hint: Recall the knowledge in linear transformation, can a single input results in multiple outputs?)\\
    Geometric: One direction cannot have 2 stretching coefficient!
\end{enumerate}

Notice that it is not necessary for a diagonalization matrix to have $n$ distinct eigenvalues.

\vspace{3pt}
More challenging: Eigenvectors that correspond to dintinct eigenvalues are linearly independent. (Hint: Can a k-dimensional space has $k+1$ stretching coefficients?)

\end{frame}

\begin{frame}{Algebraic Multiplicity and Geometric Multiplicity}
Can we find a general method to determine whether a matrix is diagonlizable or not? The method is very simple, find all the eigenvectors and count if there are $n$ independent eigenvectors.

\begin{example}
    Decide whether the following matrix $A$ is diagonlizable or not.
    \begin{equation*}
        A=\left[ \begin{matrix}
            5&		-1&		-1\\
            3&		1&		-1\\
            4&		-2&		1\\
        \end{matrix} \right]
    \end{equation*}
\end{example}
\begin{equation*}
    \det \left( A-\lambda I \right) =-\left( \lambda -2 \right) ^2\left( \lambda -3 \right)=0
\end{equation*}
For a $n$ degree polynomial, we can find $n$ roots.
\begin{equation*}
    \lambda _1=\lambda _2=2,\lambda _3=3
\end{equation*}
We call that $2$ is a repeated eigenvalue, with algebraic multiplicity of $2$.
\end{frame}

\begin{frame}{Algebraic Multiplicity and Geometric Multiplicity}
\begin{example}
    Decide whether the following matrix $A$ is diagonlizable or not.
    \begin{equation*}
        A=\left[ \begin{matrix}
            5&		-1&		-1\\
            3&		1&		-1\\
            4&		-2&		1\\
        \end{matrix} \right]
    \end{equation*}
\end{example}
\begin{equation*}
    \lambda _1=\lambda _2=2,\lambda _3=3
\end{equation*}
For those repeated eigenvalues, we need to check whether it has sufficient independent eigenvectors (for those not repeated ones, we can skip because it must have 1 independent eigenvector). Back to chapter 2, Find the nullspace dimension of $A-\lambda I$!
\end{frame}







\end{document}