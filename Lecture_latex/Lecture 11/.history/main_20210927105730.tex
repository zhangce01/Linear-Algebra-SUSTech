\documentclass{beamer}
\usepackage[utf8]{inputenc}

\usetheme{Madrid}
\usecolortheme{default}
\useinnertheme{circles}

\definecolor{Logo1}{rgb}{0.208, 0.2865, 0.373}
\definecolor{Logo2}{rgb}{0.000, 0.674, 0.863}

\setbeamercolor*{palette primary}{bg=Logo1, fg=white}
\setbeamercolor*{palette secondary}{bg=Logo2, fg=white}
\setbeamercolor*{palette tertiary}{bg=white, fg=Logo1}
\setbeamercolor*{palette quaternary}{bg=Logo1,fg=white}
\setbeamercolor{structure}{fg=Logo1} % itemize, enumerate, etc
\setbeamercolor{section in toc}{fg=Logo1} % TOC sections

\usepackage{graphicx,animate}
%------------------------------------------------------------
%This block of code defines the information to appear in the
%Title page
\title[Linear Algebra] %optional
{Row Exchanges; Inverses and Vector Spaces}

\subtitle{Lecture 2}

\author[11910803@mail.sustech.edu.cn] % (optional)
{
    Zhang Ce
}

\institute[] % (optional)
{
    Department of Electrical and Electronic Engineering\\
    Southern University of Science and Technology
}

\date[2021.9.28] % (optional)
{2021.9.28}


%End of title page configuration block
%------------------------------------------------------------



%------------------------------------------------------------
%The next block of commands puts the table of contents at the
%beginning of each section and highlights the current section:

\AtBeginSection[]
{
\begin{frame}
    \frametitle{Table of Contents}
    \tableofcontents[currentsection]
\end{frame}
}
%------------------------------------------------------------


\begin{document}

%The next statement creates the title page.
\frame{\titlepage}


%---------------------------------------------------------
%This block of code is for the table of contents after
%the title page
\begin{frame}
\frametitle{Table of Contents}
\tableofcontents
\end{frame}
%---------------------------------------------------------

\section{Row Exchanges and $PA=LU$ Factorization}

\begin{frame}{Singular Cases of Gauss Elimination}
Recall that there are 2 singular cases when using the method of Gauss elimination, they can not be expressed as $A=LU$ factorization!
\end{frame}







\section{The Geometry of Linear Equations}
\begin{frame}{Linear Equations}


\begin{examples}
Consider this linear equations system with 2 unknowns and 2 equations,
\begin{equation*}
    \begin{cases}
	2x-y=0\\
	-x+2y=3\\
\end{cases}
\end{equation*}
Solve the values of $x$, $y$.
\end{examples}

\textbf{Equivalent Matrix Form:}

\begin{equation*}
    \left[ \begin{matrix}
	2&		-1\\
	-1&		2\\
\end{matrix} \right] \left[ \begin{array}{c}
	x\\
	y\\
\end{array} \right] =\left[ \begin{array}{c}
	0\\
	3\\
\end{array} \right]
\end{equation*}

\begin{equation*}
    A\mathbf{x}=\mathbf{b}
\end{equation*}

Try to understand it in geometric way.
\end{frame}

\begin{frame}{Row Picture}
\vspace{-5pt}
\begin{equation*}
    \begin{cases}
	2x-y=0\\
	-x+2y=3\\
\end{cases}
\end{equation*}

Every row (equation) represents a single straight line.

The problem is to find the intersection of those straight lines.

\begin{figure}
    \centering
    \includegraphics[width=0.55\textwidth]{Row.jpg}
\end{figure}

\end{frame}

\begin{frame}{Column Picture}
\begin{columns}
\column{0.5\textwidth}
\begin{equation*}
    \begin{cases}
	2x-y=0\\
	-x+2y=3\\
\end{cases}
\end{equation*}

\column{0.5\textwidth}
\vspace{-9pt}
\begin{equation*}
    x\left[ \begin{array}{c}
	2\\
	-1\\
\end{array} \right] +y\left[ \begin{array}{c}
	-1\\
	2\\
\end{array} \right] =\left[ \begin{array}{c}
	0\\
	3\\
\end{array} \right]
\end{equation*}
\end{columns}

\vspace{4pt}
Every column (coefficients of a single variable) represents a vector.

The problem is to find the \textbf{Linear Combination} of those vectors.
\vspace{-1.03pt}

\begin{figure}
    \centering
    \includegraphics[width=0.48\textwidth]{Column.jpg}
\end{figure}

\end{frame}

\section{Gaussian Elimination and Back-Substitution}
\begin{frame}{Gaussian Elimination}
The core problem of linear algebra is to solve linear equations.

Consider the following system of linear equations:

\begin{equation*}
    \begin{cases}
        x+2y+z=2\\
        3x+8y+z=12\\
        4y+z=2\\
    \end{cases}
\end{equation*}

Augmented matrix:
\begin{equation*}
    \left[ \begin{matrix}
        1&		2&		1&		2\\
        3&		8&		1&		12\\
        0&		4&		1&		2\\
    \end{matrix} \right]
\end{equation*}

How can we solve it?
\end{frame}

\begin{frame}{Gaussian Elimination}
Eliminate entry on position $(2,1)$:
\begin{equation*}
    \left[ \begin{matrix}
        1&		2&		1&		2\\
        0&		2&		-2&		6\\
        0&		4&		1&		2\\
    \end{matrix} \right]
\end{equation*}

Eliminate entry on position $(3,1)$:
\begin{equation*}
    (0, skip \: this \: step)
\end{equation*}

Eliminate entry on position $(3,2)$:
\begin{equation*}
    \left[ \begin{matrix}
        1&		2&		1&		2\\
        0&		2&		-2&		6\\
        0&		0&		5&		-10\\
    \end{matrix} \right]
\end{equation*}

Elimination process done! Find the pivots above.
\end{frame}


\begin{frame}{Back-Substitution}
After elimination, we get
\begin{columns}
    \column{0.5\textwidth}
    \begin{equation*}
        \left[ \begin{matrix}
            1&		2&		1&		2\\
            0&		2&		-2&		6\\
            0&		0&		5&		-10\\
        \end{matrix} \right]
    \end{equation*}

    \column{0.5\textwidth}
    \begin{equation*}
        \begin{cases}
            x+2y+z=2\\
            2y-2z=6\\
            5z=-10\\
        \end{cases}
    \end{equation*}
\end{columns}
Do back-substitution, we can get the solution:
\begin{equation*}
    \begin{cases}
        x=2\\
        y=1\\
        z=-2\\
    \end{cases}
\end{equation*}

Rethink the whole elimination process, are there exist some cases to let the elimination process break down?
\end{frame}

\begin{frame}{Singular Cases for Gauss Elimination}
The matrix we used above is a "good" matrix.
\begin{equation*}
    \left[ \begin{matrix}
        1&		2&		1&		2\\
        3&		8&		1&		12\\
        0&		4&		1&		2\\
    \end{matrix} \right]
\end{equation*}

If we slightly change one of the element\dots
\begin{equation*}
    \left[ \begin{matrix}
        1&		2&		1&		2\\
        3&		6&		1&		12\\
        0&		4&		1&		2\\
    \end{matrix} \right]
\end{equation*}

What will happen now?
\end{frame}

\begin{frame}{Singular Cases for Gauss Elimination}
Eliminate entry on position $(2,1)$:
\begin{equation*}
    \left[ \begin{matrix}
        1&		2&		1&		2\\
        0&		0&		-2&		6\\
        0&		4&		1&		2\\
    \end{matrix} \right]
\end{equation*}

Eliminate entry on position $(3,1)$:
\begin{equation*}
    (0, skip \: this \: step)
\end{equation*}

Eliminate entry on position $(3,2)$:
\begin{equation*}
    Something \: strange \: happens.
\end{equation*}

Elimination process fails! Row exchange to fix.

\end{frame}

\begin{frame}{Singular Cases for Gauss Elimination}
Another change to make the "good" matrix singular:
\begin{equation*}
    \left[ \begin{matrix}
        1&		2&		1&		2\\
        3&		8&		1&		12\\
        0&		4&		-4&		2\\
    \end{matrix} \right]
\end{equation*}

Eliminate entry on position $(2,1)$:
\begin{equation*}
    \left[ \begin{matrix}
        1&		2&		1&		2\\
        0&		2&		-2&		6\\
        0&		4&		-4&		2\\
    \end{matrix} \right]
\end{equation*}

Eliminate entry on position $(3,2)$:
\begin{equation*}
    \left[ \begin{matrix}
        1&		2&		1&		2\\
        0&		2&		-2&		6\\
        0&		0&		0&		-10\\
    \end{matrix} \right]
\end{equation*}

Elimination process fails! Missing pivots. No method to fix.

\end{frame}

\begin{frame}{Singular Cases for Gauss Elimination}
Two kind of singular cases that can let Gauss elimination break down:
\begin{itemize}
    \item Temporal failure: Can be fixed by row exchange.
    \item Permanent failure: Missing pivots. No method to fix.
\end{itemize}

\vspace{7pt}
Recall the problem you meet in assignment:

\vspace{5pt}
Which three numbers of $k$ does elimination break down?
\begin{equation*}
    \begin{cases}
        kx+3y=6\\
        3x+ky=-6\\
    \end{cases}
\end{equation*}

$k=0$ laeds to temporal failure, while $k=3\:or\:k=-3$ leads to permanent failure.

\end{frame}

\section{Matrix Multiplication}
\begin{frame}{Matrix Size in Matrix Multiplication}
Matrix $A$: $(m \times n)$, Matrix $B$: $(n \times p)$.

\vspace{3pt}
The size of $AB$: $(m \times p)$.

\vspace{3pt}
Consider the following multiplications, are they legal?

\begin{equation*}
    \left[ \begin{array}{c}
        1\\
        2\\
        5\\
    \end{array} \right] \left[ \begin{matrix}
        1&		2&		5\\
    \end{matrix} \right] , \left[ \begin{matrix}
        1&		2&		5\\
    \end{matrix} \right] \left[ \begin{array}{c}
        1\\
        2\\
        5\\
    \end{array} \right] , \left[ \begin{matrix}
        1&		2\\
        3&		3\\
        2&		1\\
    \end{matrix} \right] \left[ \begin{array}{c}
        1\\
        2\\
        3\\
    \end{array} \right]
\end{equation*}

\begin{equation*}
    \left[ \begin{matrix}
        1&		2\\
        3&		3\\
        2&		1\\
    \end{matrix} \right] \left[ \begin{array}{c}
        3\\
        7\\
    \end{array} \right] , \left[ \begin{matrix}
        3&		3\\
        5&		4\\
    \end{matrix} \right] \left[ \begin{matrix}
        4&		5&		6&		2\\
        3&		2&		4&		3\\
    \end{matrix} \right], \left[ 5 \right] \left[ \begin{array}{c}
        3\\
        2\\
        1\\
    \end{array} \right]
\end{equation*}

The size of the multiplication result will be...

Now, I will introduce 4 methods to calculate matrix multiplication.
\end{frame}

\begin{frame}{Matrix Multiplication - 4 Methods}
\begin{examples}
Compute the matrix multiplication.
    \begin{equation*}
        \left[ \begin{matrix}
            3&		-1&       1\\
            1&		 0&       1\\
            0&		 1&       2\\
        \end{matrix} \right] \left[ \begin{matrix}
            1&		 0&       1\\
            1&		-1&      -1\\
            1&		 2&       1\\
        \end{matrix} \right] =\left[ \begin{matrix}
            3&		 3&       5\\
            2&		 2&       2\\
            3&		 3&       1\\
        \end{matrix} \right]
    \end{equation*}
\end{examples}




\begin{enumerate}
    \item The regular (row-col) way. (Row of A) multiply (Column of B).
    \item The row way. Linear combination of (Row of B).
    \item The column way. Linear combination of (Column of A).
    \item The col-row way. (Column of A) multiply (Row of B).
\end{enumerate}

There is also an additional method: block method.

\vspace{3pt}
When you calculate matrix multiplication in the exam, please remember to choose another method for verification.
\end{frame}

\section{LU Factorization}
\begin{frame}{Elimination Matrices}
Now it's time to understand elimination matrices. The goal is to express the elimination process by matrix language.

\vspace{3pt}
For example, to eliminate $(2,1)$ position element:
\begin{equation*}
    \left[ \begin{matrix}
        1&		2&		1\\
        3&		8&		1\\
        0&		4&		1\\
    \end{matrix} \right] \rightarrow \left[ \begin{matrix}
        1&		2&		1\\
        0&		2&		-2\\
        0&		4&		1\\
    \end{matrix} \right]
\end{equation*}

What we have done: (Row 2) - 3 (Row 1).

\vspace{3pt}
Use a elimination matrix $E_{21}$ to represent this process:

\begin{equation*}
    \left[ \begin{matrix}
        1&		0&		0\\
        -3&		1&		0\\
        0&		0&		1\\
    \end{matrix} \right] \left[ \begin{matrix}
        1&		2&		1\\
        3&		8&		1\\
        0&		4&		1\\
    \end{matrix} \right] =\left[ \begin{matrix}
        1&		2&		1\\
        0&		2&		-2\\
        0&		4&		1\\
    \end{matrix} \right]
\end{equation*}

\begin{equation*}
    E_{21}A=A'
\end{equation*}

As you might discover, $E_{32}E_{31}E_{21}A=U$.
\end{frame}

\begin{frame}{Gauss Elimination Represented in Elimination Matrices}
The process of Gauss Elimination we introduced before:
\begin{equation*}
    \left[ \begin{matrix}
        1&		2&		1\\
        3&		8&		1\\
        0&		4&		1\\
    \end{matrix} \right] \xrightarrow{r2-3r1}\left[ \begin{matrix}
        1&		2&		1\\
        0&		2&		-2\\
        0&		4&		1\\
    \end{matrix} \right] \rightarrow \left[ \begin{matrix}
        1&		2&		1\\
        0&		2&		-2\\
        0&		4&		1\\
    \end{matrix} \right] \xrightarrow{r3-2r2}\left[ \begin{matrix}
        1&		2&		1\\
        0&		2&		-2\\
        0&		0&		5\\
    \end{matrix} \right]
\end{equation*}

Gauss Elimination represented in multipling elimination matrices:
\begin{equation*}
    \left[ \begin{matrix}
        1&		0&		0\\
        0&		1&		0\\
        0&		-2&		1\\
    \end{matrix} \right] \left[ \begin{matrix}
        1&		0&		0\\
        0&		1&		0\\
        0&		0&		1\\
    \end{matrix} \right] \left[ \begin{matrix}
        1&		0&		0\\
        -3&		1&		0\\
        0&		0&		1\\
    \end{matrix} \right] \left[ \begin{matrix}
        1&		2&		1\\
        3&		8&		1\\
        0&		4&		1\\
    \end{matrix} \right] =\left[ \begin{matrix}
        1&		2&		1\\
        0&		2&		-2\\
        0&		0&		5\\
    \end{matrix} \right]
\end{equation*}

We define $E=E_{32}E_{31}E_{21}$, then we can get $EA=U$.

\vspace{3pt}
Finally we can get $A=LU$, $L=E^{-1}={E_{21}}^{-1}{E_{31}}^{-1}{E_{32}}^{-1}$.

\vspace{3pt}
Remind that $E_{21},E_{31},...,L$ are all lower triangular matrices.
\end{frame}

\begin{frame}{Why $A=LU$, not $EA=U$ ?}
Have you ever thought about why we need to do $LU$ factorization instead of finding the overall elimination matrix $E$?

A simple comparison using the example above:

\begin{equation*}
    E=\left[ \begin{matrix}
        1&		0&		0\\
        0&		1&		0\\
        0&		-2&		1\\
    \end{matrix} \right] \left[ \begin{matrix}
        1&		0&		0\\
        0&		1&		0\\
        0&		0&		1\\
    \end{matrix} \right] \left[ \begin{matrix}
        1&		0&		0\\
        -3&		1&		0\\
        0&		0&		1\\
    \end{matrix} \right] =\left[ \begin{matrix}
        1&		0&		0\\
        -3&		1&		0\\
        6&		-2&		1\\
    \end{matrix} \right]
\end{equation*}
\begin{equation*}
    L=\left[ \begin{matrix}
        1&		0&		0\\
        3&		1&		0\\
        0&		0&		1\\
    \end{matrix} \right] \left[ \begin{matrix}
        1&		0&		0\\
        0&		1&		0\\
        0&		0&		1\\
    \end{matrix} \right] \left[ \begin{matrix}
        1&		0&		0\\
        0&		1&		0\\
        0&		2&		1\\
    \end{matrix} \right] =\left[ \begin{matrix}
        1&		0&		0\\
        3&		1&		0\\
        0&		2&		1\\
    \end{matrix} \right]
\end{equation*}

Computing $E$ is a quite complicated thing, on the contrary, when computing $L$, we just need to fill the blank and do not need any kinds of matrix multiplication!
\end{frame}

\begin{frame}{LU Factorization}
\begin{examples}
    Do the $LU$ factorization for matrix $A$:
    \begin{equation*}
        A=\left[ \begin{matrix}
            1&		1&		1\\
            1&		4&		4\\
            1&		4&		8\\
        \end{matrix} \right]
    \end{equation*}
\end{examples}

\textbf{Solution:}
\begin{equation*}
    \left[ \begin{matrix}
        1&		1&		1\\
        1&		4&		4\\
        1&		4&		8\\
    \end{matrix} \right] \xrightarrow{l_{21}=1}\left[ \begin{matrix}
        1&		1&		1\\
        0&		3&		3\\
        1&		4&		8\\
    \end{matrix} \right] \xrightarrow{l_{31}=1}\left[ \begin{matrix}
        1&		1&		1\\
        0&		3&		3\\
        0&		3&		7\\
    \end{matrix} \right] \xrightarrow{l_{32}=1}\left[ \begin{matrix}
        1&		1&		1\\
        0&		3&		3\\
        0&		0&		4\\
    \end{matrix} \right]
\end{equation*}

\begin{equation*}
    A=\left[ \begin{matrix}
        1&		1&		1\\
        1&		4&		4\\
        1&		4&		8\\
    \end{matrix} \right] =\left[ \begin{matrix}
        1&		0&		0\\
        1&		1&		0\\
        1&		1&		1\\
    \end{matrix} \right] \left[ \begin{matrix}
        1&		1&		1\\
        0&		3&		3\\
        0&		0&		4\\
    \end{matrix} \right]=LU
\end{equation*}
\end{frame}

\begin{frame}{LDU Factorization}
For $A$ above, continue to do LDU Factorization. That is to factor $U$ into $DU$.
One thing to remember is that $D$ is made by the pivots on the diagonal. So for the matrix $U$ above,
\begin{equation*}
    D=\left[ \begin{matrix}
        1&		0&		0\\
        0&		3&		0\\
        0&		0&		4\\
    \end{matrix} \right]
\end{equation*}

We do row operations to get matrix $L$, then we need to do column operations to get matrix $U$. The goal of this step is to do column operations to eliminate entries not on the diagonal.
\begin{equation*}
    \left[ \begin{matrix}
        1&		1&		1\\
        0&		3&		3\\
        0&		0&		4\\
    \end{matrix} \right] \xrightarrow[u_{12}=1]{c2-c1}\left[ \begin{matrix}
        1&		0&		1\\
        0&		3&		3\\
        0&		0&		4\\
    \end{matrix} \right] \xrightarrow[u_{13}=1]{c3-c1}\left[ \begin{matrix}
        1&		0&		0\\
        0&		3&		3\\
        0&		0&		4\\
    \end{matrix} \right] \xrightarrow[u_{23}=1]{c3-c2}\left[ \begin{matrix}
        1&		0&		0\\
        0&		3&		0\\
        0&		0&		4\\
    \end{matrix} \right]
\end{equation*}
\vspace{-1pt}
\begin{equation*}
    A=\left[ \begin{matrix}
        1&		1&		1\\
        1&		4&		4\\
        1&		4&		8\\
    \end{matrix} \right] =\left[ \begin{matrix}
        1&		0&		0\\
        1&		1&		0\\
        1&		1&		1\\
    \end{matrix} \right] \left[ \begin{matrix}
        1&		0&		0\\
        0&		3&		0\\
        0&		0&		4\\
    \end{matrix} \right] \left[ \begin{matrix}
        1&		1&		1\\
        0&		1&		1\\
        0&		0&		1\\
    \end{matrix} \right] =LDU=LDL^T
\end{equation*}

\end{frame}
\end{document}