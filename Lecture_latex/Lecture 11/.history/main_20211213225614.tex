\documentclass{beamer}
\usepackage[utf8]{inputenc}

\usetheme{Madrid}
\usecolortheme{default}
\useinnertheme{circles}

\definecolor{Logo1}{rgb}{0.208, 0.2865, 0.373}
\definecolor{Logo2}{rgb}{0.000, 0.674, 0.863}

\setbeamercolor*{palette primary}{bg=Logo1, fg=white}
\setbeamercolor*{palette secondary}{bg=Logo2, fg=white}
\setbeamercolor*{palette tertiary}{bg=white, fg=Logo1}
\setbeamercolor*{palette quaternary}{bg=Logo1,fg=white}
\setbeamercolor{structure}{fg=Logo1} % itemize, enumerate, etc
\setbeamercolor{section in toc}{fg=Logo1} % TOC sections

\usepackage{graphicx,animate}
%------------------------------------------------------------
%This block of code defines the information to appear in the
%Title page
\title[Linear Algebra] %optional
{Similarity Transformations; Exercise Problems}

\subtitle{Lecture 12}

\author[11910803@mail.sustech.edu.cn] % (optional)
{
    Zhang Ce
}

\institute[] % (optional)
{
    Department of Electrical and Electronic Engineering\\
    Southern University of Science and Technology
}

\date[2021.12.14] % (optional)
{2021.12.14}


%End of title page configuration block
%------------------------------------------------------------



%------------------------------------------------------------
%The next block of commands puts the table of contents at the
%beginning of each section and highlights the current section:

\AtBeginSection[]
{
\begin{frame}
    \frametitle{Table of Contents}
    \tableofcontents[currentsection]
\end{frame}
}
%------------------------------------------------------------


\begin{document}

%The next statement creates the title page.
\frame{\titlepage}


%---------------------------------------------------------
%This block of code is for the table of contents after
%the title page
\begin{frame}
\frametitle{Table of Contents}
\tableofcontents
\end{frame}
%---------------------------------------------------------
\section{A Brief Review of Last Lecture}
\begin{frame}{Last Lecture, We Discuss\dots}
Two parts in last lecture:
    \begin{enumerate}
        \item Matrix Diagonaliztion\\
        simplified criteria; algebraic multiplicity and geometric multiplicity; determine a matrix is diagonalizable or not
        \item Complex Matrix\\
        introduction; Hermitian; dot products; Hermitian matrix; unitary matrix; orthogonal diagonalization for real symmetric matrix; unitary diagonaliztion for Hermitian matrix; Gram-Schmidt
    \end{enumerate}

\end{frame}

\begin{frame}{Simplified Criteria for Matrix Diagonalization}
If a $n\times n$ matrix has $n$ distinct eigenvalues, then it must be diagonalizable.

\vspace{3pt}
Why? I will not give you the proof, but I want you to understand. I will show from 2 sides:

\begin{enumerate}
    \item For each eigenvalue, there must be at least 1 eigenvector (authentic to say: 1-dimensional).
    \item For the same eigenvector (all vectors that in the 1-dimensional line), it cannot correspond to 2 eigenvalues.\\
    (Hint: Recall the knowledge in linear transformation, can a single input results in multiple outputs?)\\
    Geometric: One direction cannot have 2 stretching coefficient!
\end{enumerate}

Notice that it is not necessary for a diagonalization matrix to have $n$ distinct eigenvalues.

\vspace{3pt}
More challenging: Eigenvectors that correspond to distinct eigenvalues are linearly independent. (Hint: Can a k-dimensional space has $k+1$ stretching coefficients?)

\end{frame}

\begin{frame}{Algebraic Multiplicity and Geometric Multiplicity}
A theorem that I don't want to prove:
\begin{center}
    Geometric Multiplicity $\leq $ Algebraic Multiplicity
\end{center}

\textbf{A Summary - Liar's Game:}
\begin{enumerate}
    \item All the eigenvalues are smart, they will only claim they have higher or equal multiplicity (algebraic multiplicity) than it actually has.
    \item Your mission is to catch the liar. You investigate the possible liars (repeated eigenvalues) by calculating the nullspace dimension of $\det (A-\lambda I)$. If you find that the eigenvalues don't have the same number of independent eigenvectors as it claimed, it is a liar. Once you catch 1 liar, the matrix is not diagonalizable any more.
\end{enumerate}

Well, that is only an analogy... That's quite similar, right?
\end{frame}

\begin{frame}{Hermitian Matrix}
For real matrix, a matrix is symmetric if $A^T=A$, so for the complex matrix, a matrix is Hermitian if $A^H=A$.

\vspace{3pt}
The following matrix is a Hermitian matrix:
\begin{equation*}
    A=\left[ \begin{matrix}
        2&		3-i\\
        3+i&		5\\
    \end{matrix} \right]
\end{equation*}

Recall that for real symmetric matrix, the eigenvectors corresponding to different eigenvalues are orthogonal. This property remains true for the Hermitian matrix.

\vspace{3pt}
Another property: Every eigenvalue of a Hermitian matrix is real. This property is definitely true for real symmetric matrix because it is a special case of Hermitian!
\end{frame}

\begin{frame}{Unitary Matrix}
For real matrix, a matrix is orthogonal if $A^TA=AA^T=I$, so for the complex matrix, a matrix is unitary if $A^HA=AA^H=I$. Unitary has orthonormal column vectors.

\vspace{5pt}
\textbf{Properties:}
\begin{itemize}
    \item Inner products and lengths are preserved by $U$.
    \begin{equation*}
        \left( Ux \right) ^H\left( Uy \right) =x^Hy, \left\| Ux \right\| =\left\| x \right\|
    \end{equation*}
    \item Every eigenvector of $U$ has absolute value $|\lambda| = 1$.
    \item Eigenvectors corresponding to different eigenvalues are orthogonal.
\end{itemize}

\vspace{3pt}
For example, consider $2\times 2$ rotation matrix, it will have orthogonal eigenvectors. Also true for permutation matrix, which is a orthogonal matrix and of course unitary.

\end{frame}

\section{Similarity Transformation}
\begin{frame}{Similar Matrices}
\begin{definition}
    Two matrices $A$ and $B$ are said to be similar if there is an invertible matrix $M$ such that
    \begin{equation*}
        B=M^{-1}AM
    \end{equation*}
    It is denoted by $A\backsim B$.
\end{definition}

For similar matrices $A$ and $B$, they must share the same eigenvalues. They have the same number of independent eigenvectors.

\begin{equation*}
    B-\lambda I=M^{-1}\left( A-\lambda I \right) M
\end{equation*}
\begin{equation*}
    \det \left( B-\lambda I \right) =\det \left( M^{-1} \right) \det \left( A-\lambda I \right) \det \left( M \right) =\det \left( A-\lambda I \right)
\end{equation*}
\begin{equation*}
    Ax=\lambda x\Rightarrow MBM^{-1}x=\lambda x\Rightarrow B\left( M^{-1}x \right) =\lambda \left( M^{-1}x \right)
\end{equation*}

\end{frame}

\begin{frame}{Properties for Similar Matrices}
    Similar matrices $A$ and $B$ have the following properties:
    \begin{itemize}
        \item $A$ has the same eigenvalues as $B$.
        \item $A$ has the same number of independent eigenvectors as $B$. (Difference: $M^{-1}$)
        \item $\det A = \det B$, $trace (A) = trace (B)$.
        \item $rank (A) = rank (B)$. (who can give me a translation?)
        \item $A$ and $B$ have the same characteristic polynomial.
    \end{itemize}

    \vspace{3pt}
    If $A$ and $B$ have the same characteristic polynomial (same eigenvalues), they are not always similar.

    \begin{equation*}
        A=\left[ \begin{matrix}
            5&		3\\
            0&		5\\
        \end{matrix} \right] , B=\left[ \begin{matrix}
            5&		0\\
            0&		5\\
        \end{matrix} \right]
    \end{equation*}


    Those properties are necessary, not sufficient (Even if you find 2 matrices that can satisfy all of them, we can not say they are similar). We will discuss later on.
\end{frame}



\begin{frame}{Similarity Transformation: Change of Basis}
Recall: matrix diagonalization $A=P \varLambda P^{-1}$, we can say $A\backsim \varLambda$.
    \begin{equation*}
        \left[ \begin{matrix}
            2&		1\\
            1&		2\\
        \end{matrix} \right] =\left[ \begin{matrix}
            1&		1\\
            -1&		1\\
        \end{matrix} \right] \left[ \begin{matrix}
            1&		0\\
            0&		3\\
        \end{matrix} \right] \left[ \begin{matrix}
            1&		1\\
            -1&		1\\
        \end{matrix} \right] ^{-1}
    \end{equation*}

\begin{figure}
    \centering
    \includegraphics[width=\textwidth]{PNP-1.jpg}
\end{figure}

Essence: change basis and simplify the linear transformation matrix.
\end{frame}


\begin{frame}{Similarity Transformation: Change of Basis}
    \begin{definition}
        Two matrices $A$ and $B$ are said to be similar if there is an invertible matrix $M$ such that
        \begin{equation*}
            B=M^{-1}AM
        \end{equation*}
        It is denoted by $A\backsim B$.
    \end{definition}

Actually, $A$ and $B$ represent the same linear transformation under different bases. $M$ is the "change of basis" matrix.

\begin{itemize}
    \item $A$ has the same eigenvalues as $B$. (The strenching coefficients are the same)
    \item $A$ has the same number of independent eigenvectors as $B$. (Difference: $M^{-1}$) (The same eigenvectors! we only change the basis through $M^{-1}$)
    \item $rank (A) = rank (B)$. (the dimension of output space are the same)
\end{itemize}

Video: \url{https://www.bilibili.com/video/BV1ys411472E?p=13}

\end{frame}

\begin{frame}{Similarity Transformation: Additional}
For 2 diagonal matrices $\varLambda_1$ and $\varLambda_2$, if they have the same eigenvalues, they are similar.

\begin{equation*}
    A=\left[ \begin{matrix}
        2020&		&		&		\\
        &		2021&		&		\\
        &		&		2021&		\\
        &		&		&		2022\\
    \end{matrix} \right] , B=\left[ \begin{matrix}
        2021&		&		&		\\
        &		2022&		&		\\
        &		&		2020&		\\
        &		&		&		2021\\
    \end{matrix} \right]
\end{equation*}

Recall that every time you do matrix diagonalization, the result is not unique. You can change the columns in $P$ to change the order of eigenvalues on the diagonal. They must be similar because they are all similar to original $A$.

\vspace{3pt}
If matrices $A$ and $B$ are diagonalizable, and they have the same eigenvalues, they are similar.

\end{frame}

\begin{frame}{Schur's Lemma}
\begin{theorem}
    For a matrix of degree $n$, there exists a unitary matrix $U$ of degree $n$ such that $U^{-1}AU=T$ is triangular. The eigenvalues of $A$ appear along the diagonal of the similar matrix $T$.
\end{theorem}

That is a theorem for all matrices. So every degree $n$ matrices can be similar to a triangular matrix.

\begin{itemize}
    \item For $A=LU$, we simplify a matrix to upper triangular without changing its nullspace.
    \item For $U^{-1}AU=T$, we simplify a matrix to triangular without changing eigenvalues and number of independent eigenvectors.
\end{itemize}

We are not required to calculate this by our own (99\%). All you need to do is to understand this theorem.
\end{frame}

\begin{frame}{Normal Matrices}
\begin{theorem}
    For a matrix of degree $n$, there exists a unitary matrix $U$ of degree $n$ such that $U^{-1}AU=T$ is triangular. The eigenvalues of $A$ appear along the diagonal of the similar matrix $T$.
\end{theorem}

For some matrics, $T=\varLambda$. For that case, the matrices are called normal.

\vspace{3pt}
Normal matrices contain:
\begin{itemize}
    \item Real symmetric; Hermitian. They have all real eigenvalues.
    \item Real skew-symmetric; skew-Hermitian. They have all imaginary eigenvalues (or zero!).
    \item Orthogonal; unitary. They have eigenvalues $|\lambda|=1$.
\end{itemize}

For normal matrices, $NN^H=N^HN$. All the 3 kinds satisfy this condition.

\end{frame}

\begin{frame}{The Jordan Form (For Advanced Students)}
    Although not all the matrices are diagonalizable, we can simplify all the matrix to Jordan form (that is the nearest matrix to diagonal), which has Jordan blocks on the diagonal.

    \vspace{3pt}
    Jordan block:
    \begin{equation*}
        J=\left[ \begin{matrix}
            \lambda&		1&		&		\\
            &		\lambda&		\ddots&		\\
            &		&		\ddots&		1\\
            &		&		&		\lambda\\
        \end{matrix} \right]
    \end{equation*}

    While if a matrix has eigenvalues $\lambda$ with algebraic multiplicity of $n$ but geometric multiplicity of 1, its simplest similar matrix $J$ will have a Jordan block like this.

    \vspace{3pt}
    Our shear matrix:
    \begin{equation*}
        A=\left[ \begin{matrix}
            1&		1\\
            0&		1\\
        \end{matrix} \right]
    \end{equation*}
    is a Jordan block. It only has 1 independent eigenvectors.
\end{frame}

\begin{frame}{The Jordan Form (For Advanced Students)}
    While if a matrix has eigenvalues $\lambda$ with algebraic multiplicity of $n$ but geometric multiplicity of 2. It will has 2 Jordan blocks for this eigenvlue in the Jordan block!

    \vspace{3pt}
    For example:
    \begin{equation*}
        A=\left[ \begin{matrix}
            2&		1&		2\\
            0&		2&		0\\
            0&		0&		2\\
        \end{matrix} \right]
    \end{equation*}
    It has 2 independent eigenvectors, so it has 2 Jordan blocks in Jordan form.
    \begin{equation*}
        J=\left[ \begin{matrix}
            2&		1&		0\\
            0&		2&		0\\
            0&		0&		2\\
        \end{matrix} \right]
    \end{equation*}

\end{frame}

\begin{frame}{The Jordan Form (For Advanced Students)}
For non-diagonalizable matrices $A$ and $B$, how to determine whether they are similar? Remind that the 5 properties for similar matrices are not sufficient.

\begin{equation*}
    A=\left[ \begin{matrix}
        2&		1&		0&		0\\
        0&		2&		1&		0\\
        0&		0&		2&		0\\
        0&		0&		0&		2\\
    \end{matrix} \right] , B=\left[ \begin{matrix}
        2&		1&		0&		0\\
        0&		2&		0&		0\\
        0&		0&		2&		1\\
        0&		0&		0&		2\\
    \end{matrix} \right]
\end{equation*}

For the example above, those matrices have the same eigenvalues ($\lambda =2$ with algebraic multiplicity 4), the same number of independent eigenvectors (2).

\end{frame}



\end{document}