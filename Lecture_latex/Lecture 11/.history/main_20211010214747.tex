\documentclass{beamer}
\usepackage[utf8]{inputenc}

\usetheme{Madrid}
\usecolortheme{default}
\useinnertheme{circles}

\definecolor{Logo1}{rgb}{0.208, 0.2865, 0.373}
\definecolor{Logo2}{rgb}{0.000, 0.674, 0.863}

\setbeamercolor*{palette primary}{bg=Logo1, fg=white}
\setbeamercolor*{palette secondary}{bg=Logo2, fg=white}
\setbeamercolor*{palette tertiary}{bg=white, fg=Logo1}
\setbeamercolor*{palette quaternary}{bg=Logo1,fg=white}
\setbeamercolor{structure}{fg=Logo1} % itemize, enumerate, etc
\setbeamercolor{section in toc}{fg=Logo1} % TOC sections

\usepackage{graphicx,animate}
%------------------------------------------------------------
%This block of code defines the information to appear in the
%Title page
\title[Linear Algebra] %optional
{Column Space \& Nullspace; Solving $Ax=0$ \& $Ax=b$}

\subtitle{Lecture 3}

\author[11910803@mail.sustech.edu.cn] % (optional)
{
    Zhang Ce
}

\institute[] % (optional)
{
    Department of Electrical and Electronic Engineering\\
    Southern University of Science and Technology
}

\date[2021.10.12] % (optional)
{2021.10.12}


%End of title page configuration block
%------------------------------------------------------------



%------------------------------------------------------------
%The next block of commands puts the table of contents at the
%beginning of each section and highlights the current section:

\AtBeginSection[]
{
\begin{frame}
    \frametitle{Table of Contents}
    \tableofcontents[currentsection]
\end{frame}
}
%------------------------------------------------------------


\begin{document}

%The next statement creates the title page.
\frame{\titlepage}


%---------------------------------------------------------
%This block of code is for the table of contents after
%the title page
\begin{frame}
\frametitle{Table of Contents}
\tableofcontents
\end{frame}
%---------------------------------------------------------
\section{A Brief Review of Last Lecture}
\begin{frame}{Last Lecture, We Discuss\dots}
Three parts in last lecture:
    \begin{enumerate}
        \item Row Exchanges and $PA=LU$ Factorization\\
        permutation matrix, process of $PA=LU$ factorization
        \item Inverse of Matrix\\
        existence of inverse, calculation of inverse, block inverse
        \item Vector Spaces and Subspaces\\
        definition of vector spaces and subspaces
    \end{enumerate}

Don't forget row method and column method for matrix multiplication. Things will get easier if you know these methods!
\end{frame}

\begin{frame}{Matrix Multiplication}
\begin{examples}
Compute the matrix multiplication.
    \begin{equation*}
        \left[ \begin{matrix}
            3&		-1&       1\\
            1&		 0&       1\\
            0&		 1&       2\\
        \end{matrix} \right] \left[ \begin{matrix}
            1&		 0&       1\\
            1&		-1&      -1\\
            1&		 2&       1\\
        \end{matrix} \right] =\left[ \begin{matrix}
            3&		 3&       5\\
            2&		 2&       2\\
            3&		 3&       1\\
        \end{matrix} \right]
    \end{equation*}
\end{examples}

\begin{enumerate}
    \item The regular (row-col) way. (Row of A) multiply (Column of B).
    \item The row way. Linear combination of (Row of B).
    \item The column way. Linear combination of (Column of A).
    \item The col-row way. (Column of A) multiply (Row of B).
\end{enumerate}

There is also an additional method: block method.
\end{frame}

\begin{frame}{Why $P^T=P^{-1}$?}
Why $P^T=P^{-1}$? A direct explanation:
\vspace{3pt}

\begin{columns}
\column{0.5\textwidth}
\begin{equation*}
    P=\left[ \begin{matrix}
        0&		1&		0&		0\\
        0&		0&		1&		0\\
        0&		0&		0&		1\\
        1&		0&		0&		0\\
    \end{matrix} \right]
\end{equation*}

\begin{itemize}
    \item 1 at $(1,2)$: (Row 2) $\rightarrow$ (Row 1).
    \item 1 at $(2,3)$: (Row 3) $\rightarrow$ (Row 2).
    \item 1 at $(3,4)$: (Row 4) $\rightarrow$ (Row 3).
    \item 1 at $(4,1)$: (Row 1) $\rightarrow$ (Row 4).
\end{itemize}

\column{0.5\textwidth}
\begin{equation*}
    P^T=\left[ \begin{matrix}
        0&		0&		0&		1\\
        1&		0&		0&		0\\
        0&		1&		0&		0\\
        0&		0&		1&		0\\
    \end{matrix} \right]
\end{equation*}

\begin{itemize}
    \item 1 at $(2,1)$: (Row 1) $\rightarrow$ (Row 2).
    \item 1 at $(3,2)$: (Row 2) $\rightarrow$ (Row 3).
    \item 1 at $(4,3)$: (Row 3) $\rightarrow$ (Row 4).
    \item 1 at $(1,4)$: (Row 4) $\rightarrow$ (Row 1).
\end{itemize}
\end{columns}

\vspace{7pt}
Exchange A and B then exchange B and A will result in exchange nothing.

\end{frame}

\begin{frame}{Existence of Inverse}
According to definition, if it has an inverse, there exist a matrix $B$ to let $AB=I$, thus, $B$ is the inverse of $A$.
\begin{equation*}
    \left[ \begin{matrix}
        1&		2\\
        3&		6\\
    \end{matrix} \right] \left[ \begin{matrix}
        x&		\cdot\\
        y&		\cdot\\
    \end{matrix} \right] =\left[ \begin{matrix}
        1&		0\\
        0&		1\\
    \end{matrix} \right]
\end{equation*}

Recall the column method of matrix multiplication:
\begin{equation*}
    x\left[ \begin{array}{c}
        1\\
        3\\
    \end{array} \right] +y\left[ \begin{array}{c}
        2\\
        6\\
    \end{array} \right] =\left[ \begin{array}{c}
        1\\
        0\\
    \end{array} \right]
\end{equation*}

It may lead you to think about column picture.

\vspace{3pt}
Linear combinations of the 2 column vectors are still on the straight line.

\vspace{3pt}
Therefore, the inverse cannot exist for this matrix $A$.

\vspace{3pt}
How about multiplying this matrix on the left side?
\end{frame}

\begin{frame}{Gauss-Jordan Method: Another Way}
\begin{example}
    Find the inverse of the following matrix:
    \begin{equation*}
        A=\left[ \begin{matrix}
            1&		2\\
            3&		7\\
        \end{matrix} \right]
    \end{equation*}
\end{example}

\textbf{Solution:}

Remember: Always do row operations.
\begin{equation*}
    \left[ \begin{matrix}
        1&		2&		1&		0\\
        3&		7&		0&		1\\
    \end{matrix} \right] \rightarrow \left[ \begin{matrix}
        1&		2&		1&		0\\
        0&		1&		-3&		1\\
    \end{matrix} \right] \rightarrow \left[ \begin{matrix}
        1&		0&		7&		-2\\
        0&		1&		-3&		1\\
    \end{matrix} \right]
\end{equation*}

Or... Column operations!
\begin{equation*}
    \left[ \begin{matrix}
        1&		2\\
        3&		7\\
        1&		0\\
        0&		1\\
    \end{matrix} \right] \rightarrow \left[ \begin{matrix}
        1&		0\\
        3&		1\\
        1&		-2\\
        0&		1\\
    \end{matrix} \right] \rightarrow \left[ \begin{matrix}
        1&		0\\
        0&		1\\
        7&		-2\\
        -3&		1\\
    \end{matrix} \right]
\end{equation*}
\end{frame}



\begin{frame}{Block Method to Find the Inverse}
\begin{example}
    Find the inverse of this matrix, given that $A,B,C$ are all invertible matrices. Please express the result in matrix form.
    \begin{equation*}
        L=\left[ \begin{matrix}
            A&		0\\
            B&		C\\
        \end{matrix} \right]
    \end{equation*}
\end{example}

\textbf{Solution:}

Core idea: Treat all block matrices as a single entry.
\begin{equation*}
    \left[ \begin{matrix}
        A&		0&		I&		0\\
        B&		C&		0&		I\\
    \end{matrix} \right] \rightarrow \left[ \begin{matrix}
        A&		0&		I&		0\\
        0&		C&		-BA^{-1}&		I\\
    \end{matrix} \right] \rightarrow \left[ \begin{matrix}
        I&		0&		A^{-1}&		0\\
        0&		I&		-C^{-1}BA^{-1}&		C^{-1}\\
    \end{matrix} \right]
\end{equation*}

\begin{equation*}
    L^{-1}=\left[ \begin{matrix}
        A^{-1}&		0\\
        -C^{-1}BA^{-1}&		C^{-1}\\
    \end{matrix} \right]
\end{equation*}

A good question to ask: why not right-multiply $C^{-1}$ in the final step?
\end{frame}

\section{Vector Spaces and Subspaces}
\begin{frame}{Vectors vs Points}
When we describe a vector, we always use arrows. But, when we describe multiple vectors, treat it as points! That is a very important concept in linear algebra.

Hope this video can help you establish this concept.

\vspace{5pt}

Source: The Essence of Linear Algebra -by \emph{3Blue1Brown} P3 04:44\\
\url{https://www.bilibili.com/video/BV1ys411472E?p=3}

\end{frame}


\begin{frame}{Subspaces of $\mathbb{R} ^2$}
According to definiton, a vector space inside $\mathbb{R} ^2$ is a subspace of $\mathbb{R} ^2$.

\vspace{3pt}
Try to decide whether the following ones are subspaces of $\mathbb{R} ^2$?

\begin{enumerate}
    \item $S_1=\left\{ \left( x,0 \right) |\: x\in \mathbb{R} \right\}$
    \item $S_2=\left\{ \left( x,y \right) |\: 4x+y=0 \:\&\: x,y\in \mathbb{R} \right\}$
    \item $S_3=\left\{ \left( x,y \right) |\: x+y=1 \:\&\: x,y\in \mathbb{R} \right\}$
    \item $S_4=\left\{ \left( x,y \right) |\: xy=0 \:\&\: x,y\in \mathbb{R} \right\}$\
    \item $S_5=S_1\cup S_2$
    \item $S_6=\left\{ \left( 0,0 \right) \right\}$
    \item $S_7=\mathbb{R} ^2$
\end{enumerate}

\vspace{3pt}
Now, try to list all the subspaces for $\mathbb{R} ^2$\dots
\end{frame}

\begin{frame}{Subspaces of $\mathbb{R} ^2$}
Subspaces of $\mathbb{R} ^2$ (3 types):
\begin{enumerate}
    \item The origin. That is the zero vector $\left[ \begin{array}{c}
        0\\
        0\\
    \end{array} \right] $.
    \item Straight lines across the origin. Can I say $\mathbb{R} ^1$?
    \vspace{8pt}
    \item  $\mathbb{R} ^2$ itself.
\end{enumerate}
\end{frame}



\end{document}