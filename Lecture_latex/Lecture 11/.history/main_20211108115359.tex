\documentclass{beamer}
\usepackage[utf8]{inputenc}

\usetheme{Madrid}
\usecolortheme{default}
\useinnertheme{circles}

\definecolor{Logo1}{rgb}{0.208, 0.2865, 0.373}
\definecolor{Logo2}{rgb}{0.000, 0.674, 0.863}

\setbeamercolor*{palette primary}{bg=Logo1, fg=white}
\setbeamercolor*{palette secondary}{bg=Logo2, fg=white}
\setbeamercolor*{palette tertiary}{bg=white, fg=Logo1}
\setbeamercolor*{palette quaternary}{bg=Logo1,fg=white}
\setbeamercolor{structure}{fg=Logo1} % itemize, enumerate, etc
\setbeamercolor{section in toc}{fg=Logo1} % TOC sections

\usepackage{graphicx,animate}
%------------------------------------------------------------
%This block of code defines the information to appear in the
%Title page
\title[Linear Algebra] %optional
{Orthogonality, and Something Interesting...}

\subtitle{Lecture 7}

\author[11910803@mail.sustech.edu.cn] % (optional)
{
    Zhang Ce
}

\institute[] % (optional)
{
    Department of Electrical and Electronic Engineering\\
    Southern University of Science and Technology
}

\date[2021.11.9] % (optional)
{2021.11.9}


%End of title page configuration block
%------------------------------------------------------------



%------------------------------------------------------------
%The next block of commands puts the table of contents at the
%beginning of each section and highlights the current section:

\AtBeginSection[]
{
\begin{frame}
    \frametitle{Table of Contents}
    \tableofcontents[currentsection]
\end{frame}
}
%------------------------------------------------------------


\begin{document}

%The next statement creates the title page.
\frame{\titlepage}


%---------------------------------------------------------
%This block of code is for the table of contents after
%the title page
\begin{frame}
\frametitle{Table of Contents}
\tableofcontents
\end{frame}
%---------------------------------------------------------
\section{Orthogonal Vectors and Subspaces}
\begin{frame}{Inner Product (Dot Product)}
Actually, you have learnt that in your senior high school...

\vspace{3pt}
If I give you 2 vectors $\left[ \begin{array}{c}
	1\\
	2\\
\end{array} \right] ,\left[ \begin{array}{c}
	3\\
	1\\
\end{array} \right]$, how to compute its inner products?

\begin{equation*}
    \left[ \begin{array}{c}
        1\\
        2\\
    \end{array} \right] \cdot \left[ \begin{array}{c}
        3\\
        1\\
    \end{array} \right] =1\times 3+2\times 1=5
\end{equation*}
\end{frame}
\end{document}