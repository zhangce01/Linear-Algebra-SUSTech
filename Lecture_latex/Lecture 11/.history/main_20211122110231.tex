\documentclass{beamer}
\usepackage[utf8]{inputenc}

\usetheme{Madrid}
\usecolortheme{default}
\useinnertheme{circles}

\definecolor{Logo1}{rgb}{0.208, 0.2865, 0.373}
\definecolor{Logo2}{rgb}{0.000, 0.674, 0.863}

\setbeamercolor*{palette primary}{bg=Logo1, fg=white}
\setbeamercolor*{palette secondary}{bg=Logo2, fg=white}
\setbeamercolor*{palette tertiary}{bg=white, fg=Logo1}
\setbeamercolor*{palette quaternary}{bg=Logo1,fg=white}
\setbeamercolor{structure}{fg=Logo1} % itemize, enumerate, etc
\setbeamercolor{section in toc}{fg=Logo1} % TOC sections

\usepackage{graphicx,animate}
%------------------------------------------------------------
%This block of code defines the information to appear in the
%Title page
\title[Linear Algebra] %optional
{Computing Techniques of Determinants}

\subtitle{Lecture 9}

\author[11910803@mail.sustech.edu.cn] % (optional)
{
    Zhang Ce
}

\institute[] % (optional)
{
    Department of Electrical and Electronic Engineering\\
    Southern University of Science and Technology
}

\date[2021.11.23] % (optional)
{2021.11.23}


%End of title page configuration block
%------------------------------------------------------------



%------------------------------------------------------------
%The next block of commands puts the table of contents at the
%beginning of each section and highlights the current section:

\AtBeginSection[]
{
\begin{frame}
    \frametitle{Table of Contents}
    \tableofcontents[currentsection]
\end{frame}
}
%------------------------------------------------------------


\begin{document}

%The next statement creates the title page.
\frame{\titlepage}


%---------------------------------------------------------
%This block of code is for the table of contents after
%the title page
\begin{frame}
\frametitle{Table of Contents}
\tableofcontents
\end{frame}
%---------------------------------------------------------
\section{A Brief Review of Last Lecture}
\begin{frame}{Last Lecture, We Discuss\dots}
Four parts in last lecture:
    \begin{enumerate}
        \item Overview to Next Half Semester\\
        deteminant; eigenvalues and eigenvectors; positive definiteness
        \item Orthonormal Vectors and Orthogonal Matrices\\
        orthonormal vectors, orthogonal matrices, convenience of orthogonal matrices
        \item Gram-Schmidt and QR Decomposition\\
        analysis of 2-D case, 3-D case, and deduction of QR decomposition
        \item Introduction and Properties of Determinant\\
        geometrical view and part of the properties and computation of determinant
    \end{enumerate}

\end{frame}

\begin{frame}{Gram-Schmidt}
\textbf{Algorithm Summary:}

\vspace{3pt}
Gram's Part:
\begin{itemize}
    \item Accept $\mathbf{a}$ to the orthogonal vector set.
    \begin{equation*}
        \mathbf{A}=\mathbf{a}
    \end{equation*}
    \item Subtract $\mathbf{A}$ component from $\mathbf{b}$ and add to the orthogonal vector set.
    \begin{equation*}
        \mathbf{B}=\mathbf{b}-\frac{\mathbf{b}^T\mathbf{A}}{\mathbf{A}^T\mathbf{A}}\mathbf{A}
    \end{equation*}
\end{itemize}

Schmidt's Part:
\begin{itemize}
    \item Normalize the vectors in orthogonal vector set.
    \begin{equation*}
        \mathbf{A}=\frac{\mathbf{A}}{||\mathbf{A}||}, \mathbf{B}=\frac{\mathbf{B}}{||\mathbf{B}||}
    \end{equation*}
\end{itemize}
\end{frame}


\begin{frame}{QR Decomposition}
Experts in Linear Algebra will not stop here, they will go ahead to find the connection between $Q$ and $A$. That is the same with $LU$ decomposition, we find the matrix $L$ after we know how to find $U$. Now, we are going to find the connection $R$.

\begin{equation*}
    A=QR=\left( QQ^T \right) A\Rightarrow R=Q^TA
\end{equation*}
\begin{equation*}
    \left[ \begin{matrix}
        |&		|&		|\\
        a&		b&		c\\
        |&		|&		|\\
    \end{matrix} \right] =\left[ \begin{matrix}
        |&		|&		|\\
        q_1&		q_2&		q_3\\
        |&		|&		|\\
    \end{matrix} \right] \left[ \begin{matrix}
        {q_1}^Ta&		{q_1}^Tb&		{q_1}^Tc\\
        {q_2}^Ta&		{q_2}^Tb&		{q_2}^Tc\\
        {q_3}^Ta&		{q_3}^Tb&		{q_3}^Tc\\
    \end{matrix} \right]
\end{equation*}

In the Gram-Schmidt process, we can guarantee that ${q_2}^Ta=0$, why?

\begin{equation*}
    \left[ \begin{matrix}
        |&		|&		|\\
        a&		b&		c\\
        |&		|&		|\\
    \end{matrix} \right] =\left[ \begin{matrix}
        |&		|&		|\\
        q_1&		q_2&		q_3\\
        |&		|&		|\\
    \end{matrix} \right] \left[ \begin{matrix}
        {q_1}^Ta&		{q_1}^Tb&		{q_1}^Tc\\
        0&		{q_2}^Tb&		{q_2}^Tc\\
        0&		0&		{q_3}^Tc\\
    \end{matrix} \right]
\end{equation*}

$R$ is upper triangular! QR decomposition complete. A little bit complex...
\end{frame}

\begin{frame}{Properties of Determinant}
Now, let's consider the following properties for determinants geometrically.
\begin{itemize}
    \item $\left| \begin{matrix}
        1&		0\\
        0&		1\\
    \end{matrix} \right|=1, \left| \begin{matrix}
        0&		1\\
        1&		0\\
    \end{matrix} \right|=-1$
    \item $\left| \begin{matrix}
        ta&		b\\
        tc&		d\\
    \end{matrix} \right|=t\left| \begin{matrix}
        a&		b\\
        c&		d\\
    \end{matrix} \right|$
    \item Linearly dependent columns make the determinant 0
    \item $\det U=\left| \begin{matrix}
        d_1&		\ast&		\ast&		\cdots&		\ast\\
        0&		d_2&		\ast&		\cdots&		\ast\\
        0&		0&		d_3&		\cdots&		\ast\\
        \vdots&		\vdots&		\vdots&		\ddots&		\ast\\
        0&		0&		0&		0&		d_n\\
    \end{matrix} \right|=d_1d_2d_3\cdots d_n$
    \item $\det AB=\left( \det A \right) \left( \det B \right)$
\end{itemize}

We know all of them without any kinds of computations!
\end{frame}

\section{Properties of Determinants}
\begin{frame}{Properties of Determinants}
The order of these properties come from MIT 18.06 (Gilbert).
\begin{enumerate}
    \item $\left| \begin{matrix}
        1&		0\\
        0&		1\\
    \end{matrix} \right|=1$
    \item $\left| \begin{matrix}
        0&		1\\
        1&		0\\
    \end{matrix} \right|=-1$, exchange 2 rows reverse the sign
    \item $\left| \begin{matrix}
        ta&		tb\\
        c&		d\\
    \end{matrix} \right|=t\left| \begin{matrix}
        a&		b\\
        c&		d\\
    \end{matrix} \right|$, $\left| \begin{matrix}
        a+a'&		b+b'\\
        c&		d\\
    \end{matrix} \right|=\left| \begin{matrix}
        a&		b\\
        c&		d\\
    \end{matrix} \right|+\left| \begin{matrix}
        a'&		b'\\
        c&		d\\
    \end{matrix} \right|$
    \item 2 equal rows $\rightarrow$ 0 determinant, easily proved by property 2
    \item Subtract k times row m from row n will not change the determinant
    \item Zero row $\rightarrow$ 0 determinant
\end{enumerate}
\end{frame}



\end{document}