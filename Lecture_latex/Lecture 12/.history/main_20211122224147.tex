\documentclass{beamer}
\usepackage[utf8]{inputenc}

\usetheme{Madrid}
\usecolortheme{default}
\useinnertheme{circles}

\definecolor{Logo1}{rgb}{0.208, 0.2865, 0.373}
\definecolor{Logo2}{rgb}{0.000, 0.674, 0.863}

\setbeamercolor*{palette primary}{bg=Logo1, fg=white}
\setbeamercolor*{palette secondary}{bg=Logo2, fg=white}
\setbeamercolor*{palette tertiary}{bg=white, fg=Logo1}
\setbeamercolor*{palette quaternary}{bg=Logo1,fg=white}
\setbeamercolor{structure}{fg=Logo1} % itemize, enumerate, etc
\setbeamercolor{section in toc}{fg=Logo1} % TOC sections

\usepackage{graphicx,animate}
%------------------------------------------------------------
%This block of code defines the information to appear in the
%Title page
\title[Linear Algebra] %optional
{Computations and Applications of Determinants}

\subtitle{Lecture 9}

\author[11910803@mail.sustech.edu.cn] % (optional)
{
    Zhang Ce
}

\institute[] % (optional)
{
    Department of Electrical and Electronic Engineering\\
    Southern University of Science and Technology
}

\date[2021.11.23] % (optional)
{2021.11.23}


%End of title page configuration block
%------------------------------------------------------------



%------------------------------------------------------------
%The next block of commands puts the table of contents at the
%beginning of each section and highlights the current section:

\AtBeginSection[]
{
\begin{frame}
    \frametitle{Table of Contents}
    \tableofcontents[currentsection]
\end{frame}
}
%------------------------------------------------------------


\begin{document}

%The next statement creates the title page.
\frame{\titlepage}


%---------------------------------------------------------
%This block of code is for the table of contents after
%the title page
\begin{frame}
\frametitle{Table of Contents}
\tableofcontents
\end{frame}
%---------------------------------------------------------
\section{A Brief Review of Last Lecture}
\begin{frame}{Last Lecture, We Discuss\dots}
Four parts in last lecture:
    \begin{enumerate}
        \item Overview to Next Half Semester\\
        deteminant; eigenvalues and eigenvectors; positive definiteness
        \item Orthonormal Vectors and Orthogonal Matrices\\
        orthonormal vectors, orthogonal matrices, convenience of orthogonal matrices
        \item Gram-Schmidt and QR Decomposition\\
        analysis of 2-D case, 3-D case, and deduction of QR decomposition
        \item Introduction and Properties of Determinant\\
        geometrical view and part of the properties and computation of determinant
    \end{enumerate}

\end{frame}

\begin{frame}{Gram-Schmidt}
\textbf{Algorithm Summary:}

\vspace{3pt}
Gram's Part:
\begin{itemize}
    \item Accept $\mathbf{a}$ to the orthogonal vector set.
    \begin{equation*}
        \mathbf{A}=\mathbf{a}
    \end{equation*}
    \item Subtract $\mathbf{A}$ component from $\mathbf{b}$ and add to the orthogonal vector set.
    \begin{equation*}
        \mathbf{B}=\mathbf{b}-\frac{\mathbf{b}^T\mathbf{A}}{\mathbf{A}^T\mathbf{A}}\mathbf{A}
    \end{equation*}
\end{itemize}

Schmidt's Part:
\begin{itemize}
    \item Normalize the vectors in orthogonal vector set.
    \begin{equation*}
        \mathbf{A}=\frac{\mathbf{A}}{||\mathbf{A}||}, \mathbf{B}=\frac{\mathbf{B}}{||\mathbf{B}||}
    \end{equation*}
\end{itemize}
\end{frame}


\begin{frame}{QR Decomposition}
Experts in Linear Algebra will not stop here, they will go ahead to find the connection between $Q$ and $A$. That is the same with $LU$ decomposition, we find the matrix $L$ after we know how to find $U$. Now, we are going to find the connection $R$.

\begin{equation*}
    A=QR=\left( QQ^T \right) A\Rightarrow R=Q^TA
\end{equation*}
\begin{equation*}
    \left[ \begin{matrix}
        |&		|&		|\\
        a&		b&		c\\
        |&		|&		|\\
    \end{matrix} \right] =\left[ \begin{matrix}
        |&		|&		|\\
        q_1&		q_2&		q_3\\
        |&		|&		|\\
    \end{matrix} \right] \left[ \begin{matrix}
        {q_1}^Ta&		{q_1}^Tb&		{q_1}^Tc\\
        {q_2}^Ta&		{q_2}^Tb&		{q_2}^Tc\\
        {q_3}^Ta&		{q_3}^Tb&		{q_3}^Tc\\
    \end{matrix} \right]
\end{equation*}

In the Gram-Schmidt process, we can guarantee that ${q_2}^Ta=0$, why?

\begin{equation*}
    \left[ \begin{matrix}
        |&		|&		|\\
        a&		b&		c\\
        |&		|&		|\\
    \end{matrix} \right] =\left[ \begin{matrix}
        |&		|&		|\\
        q_1&		q_2&		q_3\\
        |&		|&		|\\
    \end{matrix} \right] \left[ \begin{matrix}
        {q_1}^Ta&		{q_1}^Tb&		{q_1}^Tc\\
        0&		{q_2}^Tb&		{q_2}^Tc\\
        0&		0&		{q_3}^Tc\\
    \end{matrix} \right]
\end{equation*}

$R$ is upper triangular! QR decomposition complete. A little bit complex...
\end{frame}

\begin{frame}{Properties of Determinant}
Now, let's consider the following properties for determinants geometrically.
\begin{itemize}
    \item $\left| \begin{matrix}
        1&		0\\
        0&		1\\
    \end{matrix} \right|=1, \left| \begin{matrix}
        0&		1\\
        1&		0\\
    \end{matrix} \right|=-1$
    \item $\left| \begin{matrix}
        ta&		b\\
        tc&		d\\
    \end{matrix} \right|=t\left| \begin{matrix}
        a&		b\\
        c&		d\\
    \end{matrix} \right|$
    \item Linearly dependent columns make the determinant 0
    \item $\det U=\left| \begin{matrix}
        d_1&		\ast&		\ast&		\cdots&		\ast\\
        0&		d_2&		\ast&		\cdots&		\ast\\
        0&		0&		d_3&		\cdots&		\ast\\
        \vdots&		\vdots&		\vdots&		\ddots&		\ast\\
        0&		0&		0&		0&		d_n\\
    \end{matrix} \right|=d_1d_2d_3\cdots d_n$
    \item $\det AB=\left( \det A \right) \left( \det B \right)$
\end{itemize}

We know all of them without any kinds of computations!
\end{frame}

\section{Properties of Determinants}
\begin{frame}{Properties of Determinants}
The order of these properties come from MIT 18.06 (Gilbert).
\begin{enumerate}
    \item $\left| \begin{matrix}
        1&		0\\
        0&		1\\
    \end{matrix} \right|=1$
    \item $\left| \begin{matrix}
        0&		1\\
        1&		0\\
    \end{matrix} \right|=-1$, exchange 2 rows reverse the sign
    \item $\left| \begin{matrix}
        ta&		tb\\
        c&		d\\
    \end{matrix} \right|=t\left| \begin{matrix}
        a&		b\\
        c&		d\\
    \end{matrix} \right|$, $\left| \begin{matrix}
        a+a'&		b+b'\\
        c&		d\\
    \end{matrix} \right|=\left| \begin{matrix}
        a&		b\\
        c&		d\\
    \end{matrix} \right|+\left| \begin{matrix}
        a'&		b'\\
        c&		d\\
    \end{matrix} \right|$
    \item 2 equal rows $\rightarrow$ 0 determinant, easily proved by property 2
    \item Subtract k times row m from row n will not change the determinant
    \item Zero row $\rightarrow$ 0 determinant
\end{enumerate}
\end{frame}

\begin{frame}{Properties of Determinants}
The order of these properties come from MIT 18.06 (Gilbert).
\begin{enumerate}
    \setcounter{enumi}{6}
    \item $\det U=\left| \begin{matrix}
        d_1&		\ast&		\ast&		\cdots&		\ast\\
        0&		d_2&		\ast&		\cdots&		\ast\\
        0&		0&		d_3&		\cdots&		\ast\\
        \vdots&		\vdots&		\vdots&		\ddots&		\ast\\
        0&		0&		0&		0&		d_n\\
    \end{matrix} \right|=d_1d_2d_3\cdots d_n$
    \item Zero det means singular, nonzero det means invertible
    \item $\det AB=\left( \det A \right) \left( \det B \right)$
    \item $\det A^T = \det A$
\end{enumerate}

Permutations can be classified to odd and even! That is the same as multiplying a permutation matrix, and permutation matrices have -1 or 1 determinant. Odd row exchanges reverse the sign, while even row exchanges do not change the sign.
\end{frame}

\section{Computations of Determinants}
\begin{frame}{Big Formula}
Up to now, we haven't introduce any of the computing formula for determinant. Can we find a general formula for all the determinants?

\vspace{3pt}
Again, start from $2\times 2$ matrix. You all know that the formula for $2\times 2$ determinant is like the following, but why?
\begin{equation*}
    \left| \begin{matrix}
        a&		b\\
        c&		d\\
    \end{matrix} \right|=ad-bc
\end{equation*}

Using linearity by row:
\begin{equation*}
    \left| \begin{matrix}
        a&		b\\
        c&		d\\
    \end{matrix} \right|=\left| \begin{matrix}
        a&		0\\
        c&		d\\
    \end{matrix} \right|+\left| \begin{matrix}
        0&		b\\
        c&		d\\
    \end{matrix} \right|=\left| \begin{matrix}
        a&		0\\
        c&		0\\
    \end{matrix} \right|+\left| \begin{matrix}
        0&		b\\
        c&		0\\
    \end{matrix} \right|+\left| \begin{matrix}
        a&		0\\
        0&		d\\
    \end{matrix} \right|+\left| \begin{matrix}
        0&		b\\
        0&		d\\
    \end{matrix} \right|
\end{equation*}

What we have done: take an entry in each row, and only the determinants without the zero column are nonzero.

\vspace{3pt}
\textbf{Inspiration:} Choose entries from each row and column, only consider the sun of those determinants, the others are all zero if we take 2 entries from a single row or column.
\end{frame}

\begin{frame}{Big Formula}
Consider $3\times 3$ case:
\begin{equation*}
    \left| \begin{matrix}
        a&		b&		c\\
        d&		e&		f\\
        g&		h&		i\\
    \end{matrix} \right|
\end{equation*}
\begin{itemize}
    \item Taking $a$, then we can take $e, i$ or $f, h$.
    \item Taking $b$, then we can take $d, i$ or $f, g$.
    \item Taking $c$, then we can take $d, h$ or $e, g$.
\end{itemize}

$2\times 2$ determinants have 2 terms, $3\times 3$ determinants have 6 terms, what about $n\times n$ determinants?

\vspace{3pt}
Taking an entry from each row: for the first row, you have $n$ choices, for the second row, you have $n-1$ choices (because you can't take the entry from the same column),...

\vspace{3pt}
So, $n\times n$ determinants have $n!$ terms.
\end{frame}

\begin{frame}{Big Formula}
\textbf{BIG FORMULA:}
\begin{equation*}
    \det A=\sum_{all\,\,combinations}{\left( \det P \right)}\,a_{1\alpha}a_{2\beta}\cdots a_{n\omega}
\end{equation*}
while $P$ is the permutation matrix that have determinant 1 or -1 (determined by the order of chosen entries).

\vspace{3pt}
Another simplified expression: $P=\left( \alpha ,\beta ,\cdots ,\omega \right)$.

\vspace{3pt}
Example: Using Big Formula to show that
\begin{equation*}
    \det U=\left| \begin{matrix}
        d_1&		\ast&		\ast&		\cdots&		\ast\\
        0&		d_2&		\ast&		\cdots&		\ast\\
        0&		0&		d_3&		\cdots&		\ast\\
        \vdots&		\vdots&		\vdots&		\ddots&		\ast\\
        0&		0&		0&		0&		d_n\\
    \end{matrix} \right|=d_1d_2d_3\cdots d_n
\end{equation*}
\end{frame}

\begin{frame}{Cofactor Formula}
Consider $3\times 3$ case:
\begin{equation*}
    \left| \begin{matrix}
        a_{11}&		a_{12}&		a_{13}\\
        a_{21}&		a_{22}&		a_{23}\\
        a_{31}&		a_{32}&		a_{33}\\
    \end{matrix} \right|=a_{11}\left| \begin{matrix}
        a_{22}&		a_{23}\\
        a_{32}&		a_{33}\\
    \end{matrix} \right|-a_{12}\left| \begin{matrix}
        a_{21}&		a_{23}\\
        a_{31}&		a_{33}\\
    \end{matrix} \right|+a_{13}\left| \begin{matrix}
        a_{21}&		a_{22}\\
        a_{31}&		a_{32}\\
    \end{matrix} \right|
\end{equation*}

\textbf{COFACTOR FORMULA:}
\begin{equation*}
    \det A=a_{11}C_{11}+a_{12}C_{12}+\cdots +a_{1n}C_{1n}
\end{equation*}

Cofactors are the determinants that eliminates a row and a column, multiplying a coefficient of 1 or -1, determined by the sum of $i, j$.
\end{frame}

\section{Applications of Determinants}
\begin{frame}{Computation of Inverses}
Cofactor matrix:
\begin{equation*}
    C=\left[ \begin{matrix}
        C_{11}&		C_{12}&		\cdots&		C_{1n}\\
        C_{21}&		C_{22}&		\cdots&		C_{2n}\\
        \vdots&		\vdots&		\ddots&		\vdots\\
        C_{n1}&		C_{n2}&		\cdots&		C_{nn}\\
    \end{matrix} \right]
\end{equation*}


A formula for all square matrices (no matter singular or not):
\begin{equation*}
    AC^T=\det A \cdot I
\end{equation*}

Noteworthy that $A^*$ is the same as $C^T$, called the adjoint matrix.

\vspace{3pt}
You'd better know how it comes... Refering to MIT 18.06 please!
\url{https://www.bilibili.com/video/BV1zx411g7gq?p=20} 07:41

\vspace{3pt}
If matrix $A$ is invertible, the inverse:
\begin{equation*}
    A^{-1}=\frac{1}{\det A}A^*
\end{equation*}

\end{frame}

\begin{frame}{Cramer's Rule}
Consider a system of linear equations $Ax=b$:
\begin{equation*}
    \left[ \begin{matrix}
        a_{11}&		a_{12}&		\cdots&		a_{1n}\\
        a_{21}&		a_{22}&		\cdots&		a_{2n}\\
        \vdots&		\vdots&		\ddots&		\vdots\\
        a_{n1}&		a_{n2}&		\cdots&		a_{nn}\\
    \end{matrix} \right] \left[ \begin{array}{c}
        x_1\\
        x_2\\
        \vdots\\
        x_n\\
    \end{array} \right] =\left[ \begin{array}{c}
        b_1\\
        b_2\\
        \vdots\\
        b_n\\
    \end{array} \right]
\end{equation*}

Cramer gives
\begin{equation*}
    x_j=\frac{\det B_j}{\det A}
\end{equation*}

where
\begin{equation*}
    B_j=\left[ \begin{matrix}
        a_{11}&		a_{12}&		\cdots&		{\color[RGB]{240, 0, 0} b_1}&		\cdots&		a_{1n}\\
        a_{21}&		a_{22}&		\cdots&		{\color[RGB]{240, 0, 0} b_2}&		\cdots&		a_{2n}\\
        \vdots&		\vdots&		&		{\color[RGB]{240, 0, 0} \vdots }&		&		\vdots\\
        a_{n1}&		a_{n2}&		\cdots&		{\color[RGB]{240, 0, 0} b_n}&		\cdots&		a_{nn}\\
    \end{matrix} \right]
\end{equation*}

For $10\times 10$ matrix, you need to find eleven $10\times 10$ determinants to find the solution. Please use Gaussian Elimination to solve linear equations.
\end{frame}

\section{Topic: Techniques for Computing Determinants}
\begin{frame}{Type 1: Tri-diagonal Matrix}
(2019 Fall Final, 12 marks) For each natural number $n\geqslant 3$, find the determinant:
\begin{equation*}
    D_n=\left| \begin{matrix}
        2&		-1&		&		&		\\
        -1&		2&		-1&		&		\\
        &		-1&		2&		\ddots&		\\
        &		&		\ddots&		\ddots&		-1\\
        &		&		&		-1&		2\\
    \end{matrix} \right|_{n\times n}
\end{equation*}

\textbf{Solution:} cofactor expansion and find the recursion formula.

\vspace{3pt}
For this example, $D_n=2D_{n-1}-D_{n-2}$.

\vspace{3pt}
The first few terms: $D_1=2, D_2=3, D_3=4$.

\vspace{3pt}
So, the answer is: $D_n=n-1$.
\end{frame}

\begin{frame}{Type 1: Tri-diagonal Matrix}
Find the determinant:
\begin{equation*}
    D_6=\left| \begin{matrix}
        1&		1&		&		&		\\
        -1&		1&		1&		&		\\
        &		-1&		1&		\ddots&		\\
        &		&		\ddots&		\ddots&		1\\
        &		&		&		-1&		1\\
    \end{matrix} \right|_{6\times 6}
\end{equation*}

\vspace{3pt}
For this example, $D_n=D_{n-1}+D_{n-2}$.

\vspace{3pt}
The first few terms: $D_1=1, D_2=2$.

\vspace{3pt}
Fibonacci series, $1,2,3,5,8,13$, so the answer is $D_6=13$.
\end{frame}

\begin{frame}{Type 2: Arrow Form Matrix}
Find the determinant:
\begin{equation*}
    \left| \begin{matrix}
        1&		1&		1&		1		\\
        1&		2&		0&		0		\\
        1&		0&		3&		0		\\
        1&		0&		0&		4		\\
    \end{matrix} \right|
\end{equation*}

\textbf{Solution:} eliminate the first row by the diagonal entries, simplify to triangular matrix.

\begin{equation*}
    \left| \begin{matrix}
        1&		1&		1&		1\\
        1&		2&		0&		0\\
        1&		0&		3&		0\\
        1&		0&		0&		4\\
    \end{matrix} \right|=\left| \begin{matrix}
        1-\frac{1}{2}-\frac{1}{3}-\frac{1}{4}&		0&		0&		0\\
        1&		2&		0&		0\\
        1&		0&		3&		0\\
        1&		0&		0&		4\\
    \end{matrix} \right|=-2
\end{equation*}
\end{frame}

\begin{frame}{Type 2: Arrow Form Matrix}
Find the determinant:
\begin{equation*}
D_n=\left| \begin{matrix}
	a_1&		a_2&		a_3&		\cdots&		a_n\\
	b_2&		1&		0&		\cdots&		0\\
	b_3&		0&		1&		\cdots&		0\\
	\vdots&		\vdots&		\vdots&		\ddots&		\vdots\\
	b_n&		0&		0&		\cdots&		1\\
\end{matrix} \right|
\end{equation*}

\begin{equation*}
    D_n\left| \begin{matrix}
        a_1&		a_2&		a_3&		\cdots&		a_n\\
        b_2&		1&		0&		\cdots&		0\\
        b_3&		0&		1&		\cdots&		0\\
        \vdots&		\vdots&		\vdots&		\ddots&		\vdots\\
        b_n&		0&		0&		\cdots&		1\\
    \end{matrix} \right|=\left| \begin{matrix}
        a_1-a_2b_2-a_3b_3-\cdots&		0&		0&		\cdots&		0\\
        b_2&		1&		0&		\cdots&		0\\
        b_3&		0&		1&		\cdots&		0\\
        \vdots&		\vdots&		\vdots&		\ddots&		\vdots\\
        b_n&		0&		0&		\cdots&		1\\
    \end{matrix} \right|
\end{equation*}
So, the answer is $D_n=a_1-\sum_{i=2}^n{a_nb_n}$.
\end{frame}

\begin{frame}{Type 3: Vandermonde Determinant and Variations}
\textbf{Vandermonde Determinant:}
\begin{equation*}
    \left| \begin{matrix}
        1&		1&		1&		\cdots&		1\\
        x_1&		x_2&		x_3&		\cdots&		x_n\\
        {x_1}^2&		{x_2}^2&		{x_3}^2&		\cdots&		{x_n}^2\\
        \vdots&		\vdots&		\vdots&		\ddots&		\vdots\\
        {x_1}^{n-1}&		{x_2}^{n-1}&		{x_3}^{n-1}&		\cdots&		{x_n}^{n-1}\\
    \end{matrix} \right|=\prod_{2\leqslant j<i\leqslant n}{\left( x_i-x_j \right)}
\end{equation*}

Proof omitted. Please refer to baidu or other search engines.

\vspace{3pt}
An example:
\begin{equation*}
    \left| \begin{matrix}
        1&		1&		1&		1\\
        1&		2&		3&		4\\
        1&		4&		9&		16\\
        1&		8&		27&		64\\
    \end{matrix} \right|=\left( 2-1 \right) \left( 3-2 \right) \left( 3-1 \right) \left( 4-3 \right) \left( 4-2 \right) \left( 4-1 \right) =12
\end{equation*}
\end{frame}

\begin{frame}{Type 3: Vandermonde Determinant and Variations}
Variation 1: First row lost.
\begin{equation*}
    \left| \begin{matrix}
        x_1&		x_2&		x_3&		\cdots&		x_n\\
        {x_1}^2&		{x_2}^2&		{x_3}^2&		\cdots&		{x_n}^2\\
        {x_1}^3&		{x_2}^3&		{x_3}^3&		\cdots&		{x_n}^3\\
        \vdots&		\vdots&		\vdots&		\ddots&		\vdots\\
        {x_1}^n&		{x_2}^n&		{x_3}^n&		\cdots&		{x_n}^n\\
    \end{matrix} \right|
\end{equation*}

\textbf{Solution:} extract $x_i$ from each column and it becomes the original.

\begin{equation*}
    \left| \begin{matrix}
        x_1&		x_2&		x_3&		\cdots&		x_n\\
        {x_1}^2&		{x_2}^2&		{x_3}^2&		\cdots&		{x_n}^2\\
        {x_1}^3&		{x_2}^3&		{x_3}^3&		\cdots&		{x_n}^3\\
        \vdots&		\vdots&		\vdots&		\ddots&		\vdots\\
        {x_1}^n&		{x_2}^n&		{x_3}^n&		\cdots&		{x_n}^n\\
    \end{matrix} \right|=x_1x_2\cdots x_n\prod_{2\leqslant j<i\leqslant n}{\left( x_i-x_j \right)}
\end{equation*}
\end{frame}

\begin{frame}{Type 3: Vandermonde Determinant and Variations}
Variation 2: Other row lost.
\begin{equation*}
    \left| \begin{matrix}
        1&		1&		1&		1\\
        a^2&		b^2&		c^2&		d^2\\
        a^3&		b^3&		c^3&		d^3\\
        a^4&		b^4&		c^4&		d^4\\
    \end{matrix} \right|
\end{equation*}

\textbf{Solution:} construct complete Vandermonde and compare coefficient.

\vspace{3pt}
Construct complete Vandermonde matrix $A$:
\begin{equation*}
    \det A=\left| \begin{matrix}
        1&		1&		1&		1&		{\color[RGB]{240, 0, 0} 1}\\
        {\color[RGB]{240, 0, 0} a}&		{\color[RGB]{240, 0, 0} b}&		{\color[RGB]{240, 0, 0} c}&		{\color[RGB]{240, 0, 0} d}&		{\color[RGB]{240, 0, 0} x}\\
        a^2&		b^2&		c^2&		d^2&		{\color[RGB]{240, 0, 0} x^2}\\
        a^3&		b^3&		c^3&		d^3&		{\color[RGB]{240, 0, 0} x^3}\\
        a^4&		b^4&		c^4&		d^4&		{\color[RGB]{240, 0, 0} x^4}\\
    \end{matrix} \right|
\end{equation*}

Now, we want to find the minor $M_{25}$ of matrix $A$.
\end{frame}

\begin{frame}{Type 3: Vandermonde Determinant and Variations}
\begin{equation*}
    \det A=\left| \begin{matrix}
        1&		1&		1&		1&		{\color[RGB]{240, 0, 0} 1}\\
        {\color[RGB]{240, 0, 0} a}&		{\color[RGB]{240, 0, 0} b}&		{\color[RGB]{240, 0, 0} c}&		{\color[RGB]{240, 0, 0} d}&		{\color[RGB]{240, 0, 0} x}\\
        a^2&		b^2&		c^2&		d^2&		{\color[RGB]{240, 0, 0} x^2}\\
        a^3&		b^3&		c^3&		d^3&		{\color[RGB]{240, 0, 0} x^3}\\
        a^4&		b^4&		c^4&		d^4&		{\color[RGB]{240, 0, 0} x^4}\\
    \end{matrix} \right|
\end{equation*}

Define constant $S=\left( d-c \right) \left( d-b \right) \left( d-a \right) \left( c-b \right) \left( c-a \right) \left( b-a \right)$.

\vspace{3pt}
Calculate the Vandermonde determinant and cofactor expansion by column $n$:
\vspace{-5pt}
\begin{equation*}
    |A|=S\left( x-d \right) \left( x-c \right) \left( x-b \right) \left( x-a \right) =C_{15}+C_{25}x+C_{35}x^2+C_{45}x^3+C_{55}x^4
\end{equation*}

Compare the coefficient of $x$: $C_{25}=\left( -abc-abd-acd-bcd \right) S$.

\vspace{3pt}
So, the original determinant is $M_{25}=-C_{25}=\left( abc+abd+acd+bcd \right) S$.
\end{frame}

\begin{frame}{Type 4: Repeated (Similar Terms) Matrix}
Find the determinant:
\begin{equation*}
    A=\left| \begin{matrix}
        1+a_1&		1&		1&		1		\\
        1&		1+a_2&		1&		1		\\
        1&		1&		1+a_3&		1		\\
        1&		1&		1&		1+a_4		\\
    \end{matrix} \right|
\end{equation*}

\textbf{Solution:} add a row or column to eliminate repeated terms.
\begin{equation*}
    \det A=\left| \begin{matrix}
        1&		0&		0&		0&		0\\
        1&		1+a_1&		1&		1&		1\\
        1&		1&		1+a_2&		1&		1\\
        1&		1&		1&		1+a_3&		1\\
        1&		1&		1&		1&		1+a_4\\
    \end{matrix} \right|=\left| \begin{matrix}
        1&		-1&		-1&		-1&		-1\\
        1&		a_1&		&		&		\\
        1&		&		a_2&		&		\\
        1&		&		&		a_3&		\\
        1&		&		&		&		a_4\\
    \end{matrix} \right|
\end{equation*}

Back to Type 2. The answer is $\left( 1+\frac{1}{a_1}+\frac{1}{a_2}+\frac{1}{a_3}+\frac{1}{a_4} \right) a_1a_2a_3a_4$.
\end{frame}

\begin{frame}{Type 4: Repeated (Similar Terms) Matrix}
Find the determinant:
\begin{equation*}
    \det A=\left| \begin{matrix}
        1+a_1&		a_1&		a_1&		a_1		\\
        a_2&		1+a_2&		a_2&		a_2		\\
        a_3&		a_3&		1+a_3&		a_3		\\
        a_4&		a_4&		a_4&		1+a_4		\\
    \end{matrix} \right|
\end{equation*}

\begin{equation*}
    \det A=\left| \begin{matrix}
        1&		0&		0&		0&		0\\
        a_1&		1+a_1&		a_1&		a_1&		a_1\\
        a_2&		a_2&		1+a_2&		a_2&		a_2\\
        a_3&		a_3&		a_3&		1+a_3&		a_3\\
        a_4&		a_4&		a_4&		a_4&		1+a_4\\
    \end{matrix} \right|=\left| \begin{matrix}
        1&		-1&		-1&		-1&		-1\\
        a_1&		1&		&		&		\\
        a_2&		&		1&		&		\\
        a_3&		&		&		1&		\\
        a_4&		&		&		&		1\\
    \end{matrix} \right|
\end{equation*}

Back to Type 2. The answer is $a_1+a_2+a_3+a_4+1$.
\end{frame}

\begin{frame}{Type 4: Repeated (Similar Terms) Matrix}
Find the determinant:
\begin{equation*}
    \det A=\left| \begin{matrix}
        a_1&		b&		b&		\cdots&		b\\
        b&		a_2&		b&		\cdots&		b\\
        b&		b&		a_3&		\cdots&		b\\
        \vdots&		\vdots&		\vdots&		\ddots&		\vdots\\
        b&		b&		b&		\cdots&		a_n\\
    \end{matrix} \right|
\end{equation*}

\begin{equation*}
    \left| \begin{matrix}
        1&		0&		0&		0&		\cdots&		0\\
        b&		a_1&		b&		b&		\cdots&		b\\
        b&		b&		a_2&		b&		\cdots&		b\\
        b&		b&		b&		a_3&		\cdots&		b\\
        \vdots&		\vdots&		\vdots&		\vdots&		\ddots&		\vdots\\
        b&		b&		b&		b&		\cdots&		a_n\\
    \end{matrix} \right|=\left| \begin{matrix}
        1&		-1&		-1&		-1&		\cdots&		-1\\
        b&		a_1-b&		&		&		\cdots&		\\
        b&		&		a_2-b&		&		\cdots&		\\
        b&		&		&		a_3-b&		\cdots&		\\
        \vdots&		\vdots&		\vdots&		\vdots&		\ddots&		\vdots\\
        b&		&		&		&		\cdots&		a_n-b\\
    \end{matrix} \right|
\end{equation*}

Back to Type 2. The answer is $a_1+a_2+a_3+a_4+1$.
\end{frame}

\begin{frame}{Type 4: Repeated (Similar Terms) Matrix}
Find the determinant:
\begin{equation*}
    \det A=\left| \begin{matrix}
        a+x_1&		a+x_2&		a+x_3\\
        a^2+{x_1}^2&		a^2+{x_2}^2&		a^2+{x_3}^2\\
        a^3+{x_1}^3&		a^3+{x_2}^3&		a^2+{x_3}^3\\
    \end{matrix} \right|
\end{equation*}

\begin{equation*}
    \det A=\left| \begin{matrix}
        1&		0&		0&		0\\
        a&		a+x_1&		a+x_2&		a+x_3\\
        a^2&		a^2+{x_1}^2&		a^2+{x_2}^2&		a^2+{x_3}^2\\
        a^3&		a^3+{x_1}^3&		a^3+{x_2}^3&		a^3+{x_3}^3\\
    \end{matrix} \right|=\left| \begin{matrix}
        1&		-1&		-1&		-1\\
        a&		x_1&		x_2&		x_3\\
        a^2&		{x_1}^2&		{x_2}^2&		{x_3}^2\\
        a^3&		{x_1}^3&		{x_2}^3&		{x_3}^3\\
    \end{matrix} \right|
\end{equation*}
\begin{equation*}
    =-\left| \begin{matrix}
        1&		1&		1&		1\\
        a&		x_1&		x_2&		x_3\\
        a^2&		{x_1}^2&		{x_2}^2&		{x_3}^2\\
        a^3&		{x_1}^3&		{x_2}^3&		{x_3}^3\\
    \end{matrix} \right|
    +\left| \begin{matrix}
        2&		0&		0&		0\\
        a&		x_1&		x_2&		x_3\\
        a^2&		{x_1}^2&		{x_2}^2&		{x_3}^2\\
        a^3&		{x_1}^3&		{x_2}^3&		{x_3}^3\\
    \end{matrix} \right|
\end{equation*}
Back to Type 3.
\end{frame}

\begin{frame}{Type 5: Circulant Matrix}
Find the determinant:
\begin{equation*}
\det A=\left| \begin{matrix}
	0&		b&		0&		a\\
	a&		0&		b&		0\\
	0&		a&		0&		b\\
	b&		0&		a&		0\\
\end{matrix} \right|
\end{equation*}

We can have cofactor expansion on column 1:
\begin{equation*}
    \det A=-a\left| \begin{matrix}
        b&		0&		a\\
        a&		0&		b\\
        0&		a&		=\\
    \end{matrix} \right|-b\left| \begin{matrix}
        b&		0&		a\\
        0&		b&		0\\
        a&		0&		b\\
    \end{matrix} \right|=\left( a^2-b^2 \right) \left| \begin{matrix}
        b&		0&		a\\
        0&		1&		0\\
        a&		0&		b\\
    \end{matrix} \right|=-\left( a^2-b^2 \right) ^2
\end{equation*}

This problem is solved, but it is not a general method for Type 5.

\end{frame}

\begin{frame}{Type 5: Circulant Matrix}
Find the determinant:
\begin{equation*}
    \det A=\left| \begin{matrix}
        x&		y&		z&		w\\
        y&		x&		w&		z\\
        z&		w&		x&		y\\
        w&		z&		y&		x\\
    \end{matrix} \right|
\end{equation*}

\textbf{Solution:} factor extraction.
\begin{equation*}
    \det A=\left| \begin{matrix}
        x&		y&		z&		w\\
        y&		x&		w&		z\\
        z&		w&		x&		y\\
        w&		z&		y&		x\\
    \end{matrix} \right|=\left( x+y+z+w \right) \left| \begin{matrix}
        1&		1&		1&		1\\
        y&		x&		w&		z\\
        z&		w&		x&		y\\
        w&		z&		y&		x\\
    \end{matrix} \right|
\end{equation*}
\begin{equation*}
    \det A=\left| \begin{matrix}
        x&		y&		z&		w\\
        y&		x&		w&		z\\
        z&		w&		x&		y\\
        w&		z&		y&		x\\
    \end{matrix} \right|=\left( x+y-z-w \right) \left| \begin{matrix}
        1&		1&		-1&		-1\\
        y&		x&		w&		z\\
        z&		w&		x&		y\\
        w&		z&		y&		x\\
    \end{matrix} \right|
\end{equation*}
\end{frame}

\begin{frame}{Type 5: Circulant Matrix}
\begin{equation*}
    \det A=\left| \begin{matrix}
        x&		y&		z&		w\\
        y&		x&		w&		z\\
        z&		w&		x&		y\\
        w&		z&		y&		x\\
    \end{matrix} \right|=\left( x+z-y-w \right) \left| \begin{matrix}
        1&		-1&		1&		-1\\
        y&		x&		w&		z\\
        z&		w&		x&		y\\
        w&		z&		y&		x\\
    \end{matrix} \right|
\end{equation*}
\begin{equation*}
    \det A=\left| \begin{matrix}
        x&		y&		z&		w\\
        y&		x&		w&		z\\
        z&		w&		x&		y\\
        w&		z&		y&		x\\
    \end{matrix} \right|=\left( x+w-y-z \right) \left| \begin{matrix}
        1&		-1&		-1&		1\\
        y&		x&		w&		z\\
        z&		w&		x&		y\\
        w&		z&		y&		x\\
    \end{matrix} \right|
\end{equation*}

So, the determinant must satisfy
\begin{equation*}
    \det A=k\left( x+y+z+w \right) \left( x+y-z-w \right) \left( x+z-y+w \right) \left( x+w-y-z \right)
\end{equation*}

Check coefficient of $x^4$ (1):
\begin{equation*}
    \det A=\left( x+y+z+w \right) \left( x+y-z-w \right) \left( x+z-y+w \right) \left( x+w-y-z \right)
\end{equation*}
\end{frame}

\begin{frame}{Type 5: Circulant Matrix}
Back to the first example:
\begin{equation*}
    \det A=\left| \begin{matrix}
        0&		b&		0&		a\\
        a&		0&		b&		0\\
        0&		a&		0&		b\\
        b&		0&		a&		0\\
    \end{matrix} \right|
    \end{equation*}

    By factor extraction:
    \begin{equation*}
        \det A=k\left( a+b \right) \left( a-b \right) \left( a+b \right) \left( a-b \right)
    \end{equation*}

    Check coefficient of $a^4$ (-1):
    \begin{equation*}
        \det A=-\left( a+b \right) \left( a-b \right) \left( a+b \right) \left( a-b \right)=-\left( a^2-b^2 \right) ^2
    \end{equation*}
\end{frame}

\end{document}