\documentclass{beamer}
\usepackage[utf8]{inputenc}

\usetheme{Madrid}
\usecolortheme{default}
\useinnertheme{circles}

\definecolor{Logo1}{rgb}{0.208, 0.2865, 0.373}
\definecolor{Logo2}{rgb}{0.000, 0.674, 0.863}

\setbeamercolor*{palette primary}{bg=Logo1, fg=white}
\setbeamercolor*{palette secondary}{bg=Logo2, fg=white}
\setbeamercolor*{palette tertiary}{bg=white, fg=Logo1}
\setbeamercolor*{palette quaternary}{bg=Logo1,fg=white}
\setbeamercolor{structure}{fg=Logo1} % itemize, enumerate, etc
\setbeamercolor{section in toc}{fg=Logo1} % TOC sections

\usepackage{graphicx,animate}
%------------------------------------------------------------
%This block of code defines the information to appear in the
%Title page
\title[Linear Algebra] %optional
{Eigenvalues and Eigenvectors}

\subtitle{Lecture 10}

\author[11910803@mail.sustech.edu.cn] % (optional)
{
    Zhang Ce
}

\institute[] % (optional)
{
    Department of Electrical and Electronic Engineering\\
    Southern University of Science and Technology
}

\date[2021.11.30] % (optional)
{2021.11.30}


%End of title page configuration block
%------------------------------------------------------------



%------------------------------------------------------------
%The next block of commands puts the table of contents at the
%beginning of each section and highlights the current section:

\AtBeginSection[]
{
\begin{frame}
    \frametitle{Table of Contents}
    \tableofcontents[currentsection]
\end{frame}
}
%------------------------------------------------------------


\begin{document}

%The next statement creates the title page.
\frame{\titlepage}


%---------------------------------------------------------
%This block of code is for the table of contents after
%the title page
\begin{frame}
\frametitle{Table of Contents}
\tableofcontents
\end{frame}
%---------------------------------------------------------
\section{A Brief Review of Last Lecture}
\begin{frame}{Last Lecture, We Discuss\dots}
Four parts in last lecture:
    \begin{enumerate}
        \item Properties of Determinants\\
        10 properties of determinants; singular matrix; transposing
        \item Computations of Determinant\\
        big formula; cofactor formula
        \item Applications of Determinant\\
        computation of inverses, Cramer's rule
        \item Topic: Techniques for Computing Determinants\\
        5 types of matrix
    \end{enumerate}

\end{frame}

\begin{frame}{Big Formula}
\textbf{BIG FORMULA:}
\begin{equation*}
    \det A=\sum_{all\,\,combinations}{\left( \det P \right)}\,a_{1\alpha}a_{2\beta}\cdots a_{n\omega}
\end{equation*}
while $P$ is the permutation matrix that have determinant 1 or -1 (determined by the order of chosen entries).

\vspace{3pt}
Another simplified expression: $P=\left( \alpha ,\beta ,\cdots ,\omega \right)$.

\end{frame}

\begin{frame}{Cofactor Formula}
Consider $3\times 3$ case:
\begin{equation*}
    \left| \begin{matrix}
        a_{11}&		a_{12}&		a_{13}\\
        a_{21}&		a_{22}&		a_{23}\\
        a_{31}&		a_{32}&		a_{33}\\
    \end{matrix} \right|=a_{11}\left| \begin{matrix}
        a_{22}&		a_{23}\\
        a_{32}&		a_{33}\\
    \end{matrix} \right|-a_{12}\left| \begin{matrix}
        a_{21}&		a_{23}\\
        a_{31}&		a_{33}\\
    \end{matrix} \right|+a_{13}\left| \begin{matrix}
        a_{21}&		a_{22}\\
        a_{31}&		a_{32}\\
    \end{matrix} \right|
\end{equation*}

\textbf{COFACTOR FORMULA:}
\begin{equation*}
    \det A=a_{11}C_{11}+a_{12}C_{12}+\cdots +a_{1n}C_{1n}
\end{equation*}

Cofactors are the determinants that eliminates a row and a column, multiplying a coefficient of 1 or -1, determined by the sum of $i, j$.
\end{frame}

\begin{frame}{Computation of Inverses}
Cofactor matrix:
\begin{equation*}
    C=\left[ \begin{matrix}
        C_{11}&		C_{12}&		\cdots&		C_{1n}\\
        C_{21}&		C_{22}&		\cdots&		C_{2n}\\
        \vdots&		\vdots&		\ddots&		\vdots\\
        C_{n1}&		C_{n2}&		\cdots&		C_{nn}\\
    \end{matrix} \right]
\end{equation*}


A formula for all square matrices (no matter singular or not):
\begin{equation*}
    AC^T=\det A \cdot I
\end{equation*}

Noteworthy that $A^*$ is the same as $C^T$, called the adjoint matrix.

\vspace{3pt}
You'd better know how it comes... Referring to MIT 18.06 please!
\url{https://www.bilibili.com/video/BV1zx411g7gq?p=20} 07:41

\vspace{3pt}
If matrix $A$ is invertible, the inverse:
\begin{equation*}
    A^{-1}=\frac{1}{\det A}A^*
\end{equation*}

\end{frame}

\begin{frame}{Cramer's Rule}
Consider a system of linear equations $Ax=b$:
\begin{equation*}
    \left[ \begin{matrix}
        a_{11}&		a_{12}&		\cdots&		a_{1n}\\
        a_{21}&		a_{22}&		\cdots&		a_{2n}\\
        \vdots&		\vdots&		\ddots&		\vdots\\
        a_{n1}&		a_{n2}&		\cdots&		a_{nn}\\
    \end{matrix} \right] \left[ \begin{array}{c}
        x_1\\
        x_2\\
        \vdots\\
        x_n\\
    \end{array} \right] =\left[ \begin{array}{c}
        b_1\\
        b_2\\
        \vdots\\
        b_n\\
    \end{array} \right]
\end{equation*}

Cramer gives
\begin{equation*}
    x_j=\frac{\det B_j}{\det A}
\end{equation*}

where
\begin{equation*}
    B_j=\left[ \begin{matrix}
        a_{11}&		a_{12}&		\cdots&		{\color[RGB]{240, 0, 0} b_1}&		\cdots&		a_{1n}\\
        a_{21}&		a_{22}&		\cdots&		{\color[RGB]{240, 0, 0} b_2}&		\cdots&		a_{2n}\\
        \vdots&		\vdots&		&		{\color[RGB]{240, 0, 0} \vdots }&		&		\vdots\\
        a_{n1}&		a_{n2}&		\cdots&		{\color[RGB]{240, 0, 0} b_n}&		\cdots&		a_{nn}\\
    \end{matrix} \right]
\end{equation*}

For $10\times 10$ matrix, you need to find eleven $10\times 10$ determinants to find the solution. Please use Gaussian Elimination to solve linear equations.
\end{frame}

\begin{frame}{Type 1: Tri-diagonal Matrix}
(2019 Fall Final, 12 marks) For each natural number $n\geqslant 3$, find the determinant:
\begin{equation*}
    D_n=\left| \begin{matrix}
        2&		-1&		&		&		\\
        -1&		2&		-1&		&		\\
        &		-1&		2&		\ddots&		\\
        &		&		\ddots&		\ddots&		-1\\
        &		&		&		-1&		2\\
    \end{matrix} \right|_{n\times n}
\end{equation*}

\textbf{Solution:} cofactor expansion and find the recursion formula.

\vspace{3pt}
For this example, $D_n=2D_{n-1}-D_{n-2}$.

\vspace{3pt}
The first few terms: $D_1=2, D_2=3, D_3=4$.

\vspace{3pt}
So, the answer is: $D_n=n+1$.
\end{frame}

\begin{frame}{Type 2: Arrow Form Matrix}
Find the determinant:
\begin{equation*}
D_n=\left| \begin{matrix}
	a_1&		a_2&		a_3&		\cdots&		a_n\\
	b_2&		1&		0&		\cdots&		0\\
	b_3&		0&		1&		\cdots&		0\\
	\vdots&		\vdots&		\vdots&		\ddots&		\vdots\\
	b_n&		0&		0&		\cdots&		1\\
\end{matrix} \right|
\end{equation*}

\begin{equation*}
    D_n\left| \begin{matrix}
        a_1&		a_2&		a_3&		\cdots&		a_n\\
        b_2&		1&		0&		\cdots&		0\\
        b_3&		0&		1&		\cdots&		0\\
        \vdots&		\vdots&		\vdots&		\ddots&		\vdots\\
        b_n&		0&		0&		\cdots&		1\\
    \end{matrix} \right|=\left| \begin{matrix}
        a_1-a_2b_2-a_3b_3-\cdots&		0&		0&		\cdots&		0\\
        b_2&		1&		0&		\cdots&		0\\
        b_3&		0&		1&		\cdots&		0\\
        \vdots&		\vdots&		\vdots&		\ddots&		\vdots\\
        b_n&		0&		0&		\cdots&		1\\
    \end{matrix} \right|
\end{equation*}
So, the answer is $D_n=a_1-\sum_{i=2}^n{a_nb_n}$.
\end{frame}

\begin{frame}{Type 3: Vandermonde Determinant and Variations}
Variation 1: First row lost.
\begin{equation*}
    \left| \begin{matrix}
        x_1&		x_2&		x_3&		\cdots&		x_n\\
        {x_1}^2&		{x_2}^2&		{x_3}^2&		\cdots&		{x_n}^2\\
        {x_1}^3&		{x_2}^3&		{x_3}^3&		\cdots&		{x_n}^3\\
        \vdots&		\vdots&		\vdots&		\ddots&		\vdots\\
        {x_1}^n&		{x_2}^n&		{x_3}^n&		\cdots&		{x_n}^n\\
    \end{matrix} \right|
\end{equation*}

\textbf{Solution:} extract $x_i$ from each column and it becomes the original.

\begin{equation*}
    \left| \begin{matrix}
        x_1&		x_2&		x_3&		\cdots&		x_n\\
        {x_1}^2&		{x_2}^2&		{x_3}^2&		\cdots&		{x_n}^2\\
        {x_1}^3&		{x_2}^3&		{x_3}^3&		\cdots&		{x_n}^3\\
        \vdots&		\vdots&		\vdots&		\ddots&		\vdots\\
        {x_1}^n&		{x_2}^n&		{x_3}^n&		\cdots&		{x_n}^n\\
    \end{matrix} \right|=x_1x_2\cdots x_n\prod_{2\leqslant j<i\leqslant n}{\left( x_i-x_j \right)}
\end{equation*}
\end{frame}

\begin{frame}{Type 3: Vandermonde Determinant and Variations}
Variation 2: Other row lost.
\begin{equation*}
    \left| \begin{matrix}
        1&		1&		1&		1\\
        a^2&		b^2&		c^2&		d^2\\
        a^3&		b^3&		c^3&		d^3\\
        a^4&		b^4&		c^4&		d^4\\
    \end{matrix} \right|
\end{equation*}

\textbf{Solution:} construct complete Vandermonde and compare coefficient.

\vspace{3pt}
Construct complete Vandermonde matrix $A$:
\begin{equation*}
    \det A=\left| \begin{matrix}
        1&		1&		1&		1&		{\color[RGB]{240, 0, 0} 1}\\
        {\color[RGB]{240, 0, 0} a}&		{\color[RGB]{240, 0, 0} b}&		{\color[RGB]{240, 0, 0} c}&		{\color[RGB]{240, 0, 0} d}&		{\color[RGB]{240, 0, 0} x}\\
        a^2&		b^2&		c^2&		d^2&		{\color[RGB]{240, 0, 0} x^2}\\
        a^3&		b^3&		c^3&		d^3&		{\color[RGB]{240, 0, 0} x^3}\\
        a^4&		b^4&		c^4&		d^4&		{\color[RGB]{240, 0, 0} x^4}\\
    \end{matrix} \right|
\end{equation*}

Now, we want to find the minor $M_{25}$ of matrix $A$.
\end{frame}

\begin{frame}{Type 3: Vandermonde Determinant and Variations}
\begin{equation*}
    \det A=\left| \begin{matrix}
        1&		1&		1&		1&		{\color[RGB]{240, 0, 0} 1}\\
        {\color[RGB]{240, 0, 0} a}&		{\color[RGB]{240, 0, 0} b}&		{\color[RGB]{240, 0, 0} c}&		{\color[RGB]{240, 0, 0} d}&		{\color[RGB]{240, 0, 0} x}\\
        a^2&		b^2&		c^2&		d^2&		{\color[RGB]{240, 0, 0} x^2}\\
        a^3&		b^3&		c^3&		d^3&		{\color[RGB]{240, 0, 0} x^3}\\
        a^4&		b^4&		c^4&		d^4&		{\color[RGB]{240, 0, 0} x^4}\\
    \end{matrix} \right|
\end{equation*}

Define constant $S=\left( d-c \right) \left( d-b \right) \left( d-a \right) \left( c-b \right) \left( c-a \right) \left( b-a \right)$.

\vspace{3pt}
Calculate the Vandermonde determinant and cofactor expansion by column $n$:
\vspace{-5pt}
\begin{equation*}
    |A|=S\left( x-d \right) \left( x-c \right) \left( x-b \right) \left( x-a \right) =C_{15}+C_{25}x+C_{35}x^2+C_{45}x^3+C_{55}x^4
\end{equation*}

Compare the coefficient of $x$: $C_{25}=\left( -abc-abd-acd-bcd \right) S$.

\vspace{3pt}
So, the original determinant is $M_{25}=-C_{25}=\left( abc+abd+acd+bcd \right) S$.
\end{frame}

\begin{frame}{Type 4: Repeated (Similar Terms) Matrix}
Find the determinant:
\begin{equation*}
    A=\left| \begin{matrix}
        1+a_1&		1&		1&		1		\\
        1&		1+a_2&		1&		1		\\
        1&		1&		1+a_3&		1		\\
        1&		1&		1&		1+a_4		\\
    \end{matrix} \right|
\end{equation*}

\textbf{Solution:} add a row or column to eliminate repeated terms.
\begin{equation*}
    \det A=\left| \begin{matrix}
        1&		0&		0&		0&		0\\
        1&		1+a_1&		1&		1&		1\\
        1&		1&		1+a_2&		1&		1\\
        1&		1&		1&		1+a_3&		1\\
        1&		1&		1&		1&		1+a_4\\
    \end{matrix} \right|=\left| \begin{matrix}
        1&		-1&		-1&		-1&		-1\\
        1&		a_1&		&		&		\\
        1&		&		a_2&		&		\\
        1&		&		&		a_3&		\\
        1&		&		&		&		a_4\\
    \end{matrix} \right|
\end{equation*}

Back to Type 2. The answer is $\left( 1+\frac{1}{a_1}+\frac{1}{a_2}+\frac{1}{a_3}+\frac{1}{a_4} \right) a_1a_2a_3a_4$.
\end{frame}

\begin{frame}{Type 5: Circulant Matrix}
Find the determinant:
\begin{equation*}
    \det A=\left| \begin{matrix}
        x&		y&		z&		w\\
        y&		x&		w&		z\\
        z&		w&		x&		y\\
        w&		z&		y&		x\\
    \end{matrix} \right|
\end{equation*}

\textbf{Solution:} factor extraction.

\vspace{3pt}
So, the determinant must satisfy
\begin{equation*}
    \det A=k\left( x+y+z+w \right) \left( x+y-z-w \right) \left( x+z-y-w \right) \left( x+w-y-z \right)
\end{equation*}

Check coefficient of $x^4$ (1):
\begin{equation*}
    \det A=\left( x+y+z+w \right) \left( x+y-z-w \right) \left( x+z-y-w \right) \left( x+w-y-z \right)
\end{equation*}
\end{frame}




\end{document}