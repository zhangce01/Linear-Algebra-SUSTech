\documentclass{beamer}
\usepackage[utf8]{inputenc}

\usetheme{Madrid}
\usecolortheme{default}
\useinnertheme{circles}

\definecolor{Logo1}{rgb}{0.208, 0.2865, 0.373}
\definecolor{Logo2}{rgb}{0.000, 0.674, 0.863}

\setbeamercolor*{palette primary}{bg=Logo1, fg=white}
\setbeamercolor*{palette secondary}{bg=Logo2, fg=white}
\setbeamercolor*{palette tertiary}{bg=white, fg=Logo1}
\setbeamercolor*{palette quaternary}{bg=Logo1,fg=white}
\setbeamercolor{structure}{fg=Logo1} % itemize, enumerate, etc
\setbeamercolor{section in toc}{fg=Logo1} % TOC sections

\usepackage{graphicx,animate}
%------------------------------------------------------------
%This block of code defines the information to appear in the
%Title page
\title[Linear Algebra] %optional
{Row Exchanges; Inverses and Vector Spaces}

\subtitle{Lecture 2}

\author[11910803@mail.sustech.edu.cn] % (optional)
{
    Zhang Ce
}

\institute[] % (optional)
{
    Department of Electrical and Electronic Engineering\\
    Southern University of Science and Technology
}

\date[2021.9.28] % (optional)
{2021.9.28}


%End of title page configuration block
%------------------------------------------------------------



%------------------------------------------------------------
%The next block of commands puts the table of contents at the
%beginning of each section and highlights the current section:

\AtBeginSection[]
{
\begin{frame}
    \frametitle{Table of Contents}
    \tableofcontents[currentsection]
\end{frame}
}
%------------------------------------------------------------


\begin{document}

%The next statement creates the title page.
\frame{\titlepage}


%---------------------------------------------------------
%This block of code is for the table of contents after
%the title page
\begin{frame}
\frametitle{Table of Contents}
\tableofcontents
\end{frame}
%---------------------------------------------------------
\section{A Brief Review of Last Lecture}
\begin{frame}{Last Lecture, We Discuss\dots}
Five parts in last lecture:
    \begin{enumerate}
        \item Introduction to the Whole Course
        \item The Geometry of Linear Equations\\
        row picture \& column picture
        \item Gaussian Elimination and Back-Substitution\\
        nonsingular \& singular cases
        \item Matrix Multiplication\\
        row-col, row, column, col-row \& block method
        \item $LU$ Factorization\\
        elimination matrices \& process of $LU$ factorization
    \end{enumerate}

Which one is the most impressive to you?
\end{frame}

\begin{frame}{Matrix Multiplication}
\begin{examples}
Compute the matrix multiplication.
    \begin{equation*}
        \left[ \begin{matrix}
            3&		-1&       1\\
            1&		 0&       1\\
            0&		 1&       2\\
        \end{matrix} \right] \left[ \begin{matrix}
            1&		 0&       1\\
            1&		-1&      -1\\
            1&		 2&       1\\
        \end{matrix} \right] =\left[ \begin{matrix}
            3&		 3&       5\\
            2&		 2&       2\\
            3&		 3&       1\\
        \end{matrix} \right]
    \end{equation*}
\end{examples}

\begin{enumerate}
    \item The regular (row-col) way. (Row of A) multiply (Column of B).
    \item The row way. Linear combination of (Row of B).
    \item The column way. Linear combination of (Column of A).
    \item The col-row way. (Column of A) multiply (Row of B).
\end{enumerate}

There is also an additional method: block method.

\vspace{3pt}
Method 2 \& 3 are my favourite, and they are also important in this lecture. Let's have a brief review.
\end{frame}


\section{Row Exchanges and PA=LU Factorization}

\begin{frame}{Singular Cases for Gauss Elimination}
Recall that there are 2 singular cases when using the method of Gauss elimination, they can not be expressed as $A=LU$ factorization! But, the temporal failure can be fixed by row exchanges.

\vspace{3pt}
Do Gauss elimination (with the help of row exchanges) for the matrix below:

\begin{equation*}
    \left[ \begin{matrix}
        1&		2&		1&		2\\
        3&		6&		1&		12\\
        0&		4&		1&		2\\
    \end{matrix} \right] \rightarrow \left[ \begin{matrix}
        1&		2&		1&		2\\
        0&		0&		-2&		6\\
        0&		4&		1&		2\\
    \end{matrix} \right] \rightarrow \left[ \begin{matrix}
        1&		2&		1&		2\\
        0&		4&		1&		2\\
        0&		0&		-2&		6\\
    \end{matrix} \right]
\end{equation*}

Write the elimination matrices:

\begin{equation*}
    \left[ \begin{matrix}
        1&		0&		0\\
        -3&		1&		0\\
        0&		0&		1\\
    \end{matrix} \right] \left[ \begin{matrix}
        1&		2&		1\\
        3&		6&		1\\
        0&		4&		1\\
    \end{matrix} \right] =\left[ \begin{matrix}
        1&		2&		1\\
        0&		0&		-2\\
        0&		4&		1\\
    \end{matrix} \right]
\end{equation*}

Review: Which rule for matrix multiplication can help you understand the row operation above?

\end{frame}

\begin{frame}{Row Exchanges in Permutation Matrix}
Step 2 is to exchange Row 2 and Row 3, still using the row method for matrix multiplication, which matrix can express the process of row exchanges?

\begin{equation*}
    A\left[ \begin{matrix}
        1&		2&		1\\
        0&		0&		-2\\
        0&		4&		1\\
    \end{matrix} \right] =\left[ \begin{matrix}
        1&		2&		1\\
        0&		4&		1\\
        0&		0&		-2\\
    \end{matrix} \right]
\end{equation*}

This matrix is to exchange the rows, so we call it permutation matrix $P$.

\begin{equation*}
    P\left[ \begin{matrix}
        1&		2&		1\\
        0&		0&		-2\\
        0&		4&		1\\
    \end{matrix} \right] =\left[ \begin{matrix}
        1&		2&		1\\
        0&		4&		1\\
        0&		0&		-2\\
    \end{matrix} \right]
\end{equation*}

\end{frame}

\begin{frame}{Row Exchanges in Permutation Matrix}
The permutation matrix $P$:

\begin{equation*}
    P=\left[ \begin{matrix}
        1&		0&		0\\
        0&		0&		1\\
        0&		1&		0\\
    \end{matrix} \right]
\end{equation*}

\begin{itemize}
    \item For "1" at position $(1,1)$: (Row 1) $\rightarrow$ (Row 1).
    \item For "1" at position $(2,3)$: (Row 3) $\rightarrow$ (Row 2).
    \item For "1" at position $(3,2)$: (Row 2) $\rightarrow$ (Row 3).
\end{itemize}

One of the properties of permutation matrix $P$ is that it can only have one "1" in each row and column. Why?

\vspace{3pt}
After row exchanges, no rows before will disappear and no rows after will be all empty.
\end{frame}

\begin{frame}{Why $P^T=P^{-1}$?}
Why $P^T=P^{-1}$? A direct explanation:
\vspace{3pt}

\begin{columns}
\column{0.5\textwidth}
\begin{equation*}
    P=\left[ \begin{matrix}
        0&		1&		0&		0\\
        0&		0&		1&		0\\
        0&		0&		0&		1\\
        1&		0&		0&		0\\
    \end{matrix} \right]
\end{equation*}

\begin{itemize}
    \item 1 at $(1,2)$: (Row 2) $\rightarrow$ (Row 1).
    \item 1 at $(2,3)$: (Row 3) $\rightarrow$ (Row 2).
    \item 1 at $(3,4)$: (Row 4) $\rightarrow$ (Row 3).
    \item 1 at $(4,1)$: (Row 1) $\rightarrow$ (Row 4).
\end{itemize}

\column{0.5\textwidth}
\begin{equation*}
    P^T=\left[ \begin{matrix}
        0&		0&		0&		1\\
        1&		0&		0&		0\\
        0&		1&		0&		0\\
        0&		0&		1&		0\\
    \end{matrix} \right]
\end{equation*}

\begin{itemize}
    \item 1 at $(2,1)$: (Row 1) $\rightarrow$ (Row 2).
    \item 1 at $(3,2)$: (Row 2) $\rightarrow$ (Row 3).
    \item 1 at $(4,3)$: (Row 3) $\rightarrow$ (Row 4).
    \item 1 at $(1,4)$: (Row 4) $\rightarrow$ (Row 1).
\end{itemize}
\end{columns}

\vspace{7pt}
Exchange A and B then exchange B and A will result in exchange nothing.

\vspace{3pt}
Start to feel the beauty of multipling matrices by row?
\end{frame}


\begin{frame}{Expressing Row Exchanges using $PA=LU$}
The process of Gauss elimination and row exchanges can be expressed as:
\begin{equation*}
    \left[ \begin{matrix}
        1&		0&		0\\
        0&		0&		1\\
        0&		1&		0\\
    \end{matrix} \right] \left[ \begin{matrix}
        1&		0&		0\\
        -3&		1&		0\\
        0&		0&		1\\
    \end{matrix} \right] \left[ \begin{matrix}
        1&		2&		1\\
        3&		6&		1\\
        0&		4&		1\\
    \end{matrix} \right] =\left[ \begin{matrix}
        1&		2&		1\\
        0&		4&		1\\
        0&		0&		-2\\
    \end{matrix} \right]
\end{equation*}

\begin{equation*}
    PE_{21}A=U
\end{equation*}

Can we do row exchanges at the beginning?
\begin{equation*}
    EPA=U
\end{equation*}

Then, left multiply $E^{-1}$ on both sides:
\begin{equation*}
    PA=LU
\end{equation*}
\end{frame}

\begin{frame}{$PA=LU$ Factorization}
\begin{example}
    Do $PA=LU$ factorization for matrix $A$:
    \begin{equation*}
        A=\left[ \begin{matrix}
            1&		2&		1\\
            3&		6&		1\\
            0&		4&		1\\
        \end{matrix} \right]
    \end{equation*}
\end{example}
\textbf{Solution:}
\begin{equation*}
    \left[ \begin{matrix}
        1&		2&		1\\
        3&		6&		1\\
        0&		4&		1\\
    \end{matrix} \right] \xrightarrow[l_{21}=3]{r2-3r1}\left[ \begin{matrix}
        1&		2&		1\\
        0&		0&		-2\\
        0&		4&		1\\
    \end{matrix} \right] \xrightarrow{r2\leftrightarrow r3}\left[ \begin{matrix}
        1&		2&		1\\
        0&		4&		1\\
        0&		0&		-2\\
    \end{matrix} \right]
\end{equation*}

Restart the process and do row exchanges firstly.
\begin{equation*}
    \left[ \begin{matrix}
        1&		2&		1\\
        3&		6&		1\\
        0&		4&		1\\
    \end{matrix} \right] \xrightarrow{r2\leftrightarrow r3}\left[ \begin{matrix}
        1&		2&		1\\
        0&		4&		1\\
        3&		6&		1\\
    \end{matrix} \right] \xrightarrow[l_{31}=3]{r3-3r1}\left[ \begin{matrix}
        1&		2&		1\\
        0&		4&		1\\
        0&		0&		-2\\
    \end{matrix} \right]
\end{equation*}
\end{frame}

\begin{frame}{$PA=LU$ Factorization}
\begin{equation*}
    \left[ \begin{matrix}
        1&		0&		0\\
        0&		0&		1\\
        0&		1&		0\\
    \end{matrix} \right] \left[ \begin{matrix}
        1&		2&		1\\
        3&		6&		1\\
        0&		4&		1\\
    \end{matrix} \right] =\left[ \begin{matrix}
        1&		0&		0\\
        0&		1&		0\\
        3&		0&		1\\
    \end{matrix} \right] \left[ \begin{matrix}
        1&		2&		1\\
        0&		4&		1\\
        0&		0&		-2\\
    \end{matrix} \right]
\end{equation*}

\begin{equation*}
    PA=LU
\end{equation*}

Remember to verify the answer! Do 2 multiplications and check if they are the same.

Suppose we do not do row exchanges at first, what will happen\dots

\begin{equation*}
    \left[ \begin{matrix}
        1&		0&		0\\
        0&		0&		1\\
        0&		1&		0\\
    \end{matrix} \right] \left[ \begin{matrix}
        1&		2&		1\\
        3&		6&		1\\
        0&		4&		1\\
    \end{matrix} \right] =\left[ \begin{matrix}
        1&		0&		0\\
        3&		1&		0\\
        0&		0&		1\\
    \end{matrix} \right] \left[ \begin{matrix}
        1&		2&		1\\
        0&		4&		1\\
        0&		0&		-2\\
    \end{matrix} \right] ???
\end{equation*}

So, if you find a matrix need to have row operations, please restart and do them first. That is the only way to avoid mistakes!
\end{frame}

\section{Inverse of Matrix}
\begin{frame}{Definition of Inverse}
\begin{theorem}
An $n \times n$ matrix $A$ is said to be invertible if there is an $n \times n$ matrix $B$ such that
\begin{equation*}
    AB=BA=I
\end{equation*}
In this case, $B$ is called an \alert{inverse} of $A$.
\end{theorem}

Have you ever thought about the non-square cases? Can non-square matrices have inverse?

No, matrix size determines that! But...
\begin{equation*}
    \left[ \begin{matrix}
        1&		2&		1\\
        0&		2&		1\\
    \end{matrix} \right] \left[ \begin{matrix}
        1&		-1\\
        0&		-1\\
        0&		3\\
    \end{matrix} \right] =\left[ \begin{matrix}
        1&		0\\
        0&		1\\
    \end{matrix} \right]
\end{equation*}

Well, non-square matrices may have left inverses and right inverses.
\end{frame}

\begin{frame}{Existence of Inverse}
A very important problem is: does a particular matrix have an inverse?

\vspace{3pt}
I say that the matrix $A$ below cannot have an inverse:
\begin{equation*}
    A=\left[ \begin{matrix}
        1&		2\\
        3&		6\\
    \end{matrix} \right]
\end{equation*}
Who can give me a reason to let me believe that?

If you have learnt something about determinant, you can find that
\begin{equation*}
    \left| \begin{matrix}
        1&		2\\
        3&		6\\
    \end{matrix} \right|=6-6=0
\end{equation*}

The determinant is 0 so that the matrix is not invertible.

\vspace{3pt}
Can we use the knowledge in Chapter 1 to explain it?
\end{frame}

\begin{frame}{Existence of Inverse}
According to definition, if it has an inverse, there exist a matrix $B$ to let $AB=I$, thus, $B$ is the inverse of $A$.
\begin{equation*}
    \left[ \begin{matrix}
        1&		2\\
        3&		6\\
    \end{matrix} \right] \left[ \begin{matrix}
        x&		\cdot\\
        y&		\cdot\\
    \end{matrix} \right] =\left[ \begin{matrix}
        1&		0\\
        0&		1\\
    \end{matrix} \right]
\end{equation*}

Recall the column method of matrix multiplication:
\begin{equation*}
    x\left[ \begin{array}{c}
        1\\
        3\\
    \end{array} \right] +y\left[ \begin{array}{c}
        2\\
        6\\
    \end{array} \right] =\left[ \begin{array}{c}
        1\\
        0\\
    \end{array} \right]
\end{equation*}

It may lead you to think about column picture.

\vspace{3pt}
Is it possible? Linear combinations of the 2 column vectors are still on the straight line!

\vspace{3pt}
Therefore, the inverse cannot exist for this matrix $A$.
\end{frame}

\begin{frame}{Existence of Inverse}
I have to say I prefer to use another interpretation to explain that.

\begin{theorem}
Suppose there is a nonzero vector $\mathbf{x}$ such that $A\mathbf{x}=0$, then $A$ cannot have an inverse.
\end{theorem}

Maybe you all know that before, but why?

For this matrix $A$, find a nonzero solution for $A\mathbf{x}=0$:
\begin{equation*}
    \left[ \begin{matrix}
        1&		2\\
        3&		6\\
    \end{matrix} \right] \left[ \begin{array}{c}
        2\\
        -1\\
    \end{array} \right] =\left[ \begin{array}{c}
        0\\
        0\\
    \end{array} \right]
\end{equation*}

Suppose there is an inverse to make $A^{-1}A=I$, multiply $A^{-1}$ on both sides:
\begin{equation*}
    A\mathbf{x}=0\rightarrow A^{-1}A\mathbf{x}=A^{-1}0\rightarrow \mathbf{x}=0
\end{equation*}

Well, but there is a nonzero solution (2,-1) for matrix $A$! So matrix $A$ cannot have an inverse.
\end{frame}

\begin{frame}{Calculation of Inverse}
Another important problem: how to find the inverse $A^{-1}$ for invertible matrix $A$?

\vspace{3pt}
Gauss-Jordan method, right? But how does it come?

\vspace{3pt}
You may think up with this method:
\begin{equation*}
    \left[ \begin{matrix}
        1&		2\\
        3&		7\\
    \end{matrix} \right] \left[ \begin{matrix}
        a&		c\\
        b&		d\\
    \end{matrix} \right] =\left[ \begin{matrix}
        1&		0\\
        0&		1\\
    \end{matrix} \right]
\end{equation*}

To find the inverse, we only need to find the 4 unknowns $a,b,c,d$.

\vspace{3pt}
Use the column method of matrix multiplication to think, actually we only need to find the solutions for the following 2 linear equation systems:

\vspace{6pt}
\begin{columns}
    \column{0.5\textwidth}
    \begin{equation*}
        \left[ \begin{matrix}
            1&		2\\
            3&		7\\
        \end{matrix} \right] \left[ \begin{array}{c}
            a\\
            b\\
        \end{array} \right] =\left[ \begin{array}{c}
            1\\
            0\\
        \end{array} \right]
    \end{equation*}

    \column{0.5\textwidth}
    \begin{equation*}
        \left[ \begin{matrix}
            1&		2\\
            3&		7\\
        \end{matrix} \right] \left[ \begin{array}{c}
            c\\
            d\\
        \end{array} \right] =\left[ \begin{array}{c}
            0\\
            1\\
        \end{array} \right]
    \end{equation*}
\end{columns}

\vspace{6pt}
Then, the problem returns to Gauss elimination!
\end{frame}

\begin{frame}{Calculation of Inverse}
Let's solve one of the systems together.
\begin{equation*}
    \left[ \begin{matrix}
        1&		2\\
        3&		7\\
    \end{matrix} \right] \left[ \begin{array}{c}
        a\\
        b\\
    \end{array} \right] =\left[ \begin{array}{c}
        1\\
        0\\
    \end{array} \right]
\end{equation*}

Augmented matrix:
\begin{equation*}
    \left[ \begin{matrix}
        1&		2&		1\\
        3&		7&		0\\
    \end{matrix} \right]
\end{equation*}

Solve this equation system by Gauss elimination:
\begin{equation*}
    \left[ \begin{matrix}
        1&		2&		1\\
        3&		7&		0\\
    \end{matrix} \right] \rightarrow \left[ \begin{matrix}
        1&		2&		1\\
        0&		1&		-3\\
    \end{matrix} \right] \rightarrow \left[ \begin{matrix}
        1&		0&		\mathbf{7}\\
        0&		1&		\mathbf{-3}\\
    \end{matrix} \right]
\end{equation*}

The solution is $\left[ \begin{array}{c}
	a\\
	b\\
\end{array} \right] =\left[ \begin{array}{c}
	7\\
	-3\\
\end{array} \right] $ as the last column displays.


\end{frame}

\begin{frame}{Calculation of Inverse}
At this time, Jordan appears, he said to Gauss: "Why not solve 2 equations at once?"
\end{frame}



\end{document}